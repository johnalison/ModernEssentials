\documentclass[paper=letter,11pt]{scrartcl}

\KOMAoptions{headinclude=true, footinclude=false}
\KOMAoptions{DIV=14, BCOR=5mm}
\KOMAoptions{numbers=noendperiod}
\KOMAoptions{parskip=half}
\addtokomafont{disposition}{\rmfamily}
\addtokomafont{part}{\LARGE}
\addtokomafont{descriptionlabel}{\rmfamily}
%\setkomafont{pageheadfoot}{\normalsize\sffamily}
\setkomafont{pagehead}{\normalsize\rmfamily}
%\setkomafont{publishers}{\normalsize\rmfamily}
\setkomafont{caption}{\normalfont\small}
\setcapindent{0pt}
\deffootnote[1em]{1em}{1em}{\textsuperscript{\thefootnotemark}\ }


\usepackage{amsmath}
\usepackage[varg]{txfonts}
\usepackage[T1]{fontenc}
\usepackage{graphicx}
\usepackage{xcolor}
\usepackage[american]{babel}
% hyperref is needed in many places, so include it here
\usepackage{hyperref}

\usepackage{xspace}
\usepackage{multirow}
\usepackage{float}


\usepackage{braket}
\usepackage{bbm}
\usepackage{relsize}
\usepackage{tcolorbox}


\def\x{\ensuremath{x}}
\def\xp{\ensuremath{x'}}
\def\t{\ensuremath{t}}
\def\tp{\ensuremath{t'}}
\def\v{\ensuremath{v}}
%\def\nus{$\nu$s}

%\def\ketY{\ensuremath{\ket {\Psi}}}
%\def\iGeV{\ensuremath{\textrm{GeV}^{-1}}}
%%\def\mp{\ensuremath{m_{\textrm{proton}}}}
%\def\rp{\ensuremath{r_{\textrm{proton}}}}
%\def\me{\ensuremath{m_{\textrm{electron}}}}
%\def\aG{\ensuremath{\alpha_G}}
%\def\rAtom{\ensuremath{r_{\textrm{atom}}}}
%\def\rNucl{\ensuremath{r_{\textrm{nucleus}}}}
%\def\GN{\ensuremath{\textrm{G}_\textrm{N}}}
%\def\ketX{\ensuremath{\ket{\vec{x}}}}
%\def\ve{\ensuremath{\vec{\epsilon}}}
%
%
%\def\ABCDMatrix{\ensuremath{\begin{pmatrix} A &  B  \\ C  & D \end{pmatrix}}}
%\def\xyprime{\ensuremath{\begin{pmatrix} x' \\ y' \end{pmatrix}}}
%\def\xyprimeT{\ensuremath{\begin{pmatrix} x' &  y' \end{pmatrix}}}
%\def\xy{\ensuremath{\begin{pmatrix} x \\ y \end{pmatrix}}}
%\def\xyT{\ensuremath{\begin{pmatrix} x & y \end{pmatrix}}}
%
%\def\IMatrix{\ensuremath{\begin{pmatrix} 0 &  1  \\ -1  & 0 \end{pmatrix}}}
%\def\IBoostMatrix{\ensuremath{\begin{pmatrix} 0 &  1  \\ 1  & 0 \end{pmatrix}}}
%\def\JThree{\ensuremath{\begin{pmatrix}    0 & -i & 0  \\ i & 0  & 0 \\ 0 & 0 & 0 \end{pmatrix}}} 
%\def\JTwo{\ensuremath{\begin{bmatrix}    0 & 0 & -i  \\ 0 & 0  & 0 \\ i & 0 & 0 \end{bmatrix}}}
%\def\JOne{\ensuremath{\begin{bmatrix}    0 & 0 & 0  \\ 0 & 0  & -i \\ 0 & i & 0 \end{bmatrix}}}
%\def\etamn{\ensuremath{\eta_{\mu\nu}}}
%\def\Lmn{\ensuremath{\Lambda^\mu_\nu}}
%\def\dmn{\ensuremath{\delta^\mu_\nu}}
%\def\wmn{\ensuremath{\omega^\mu_\nu}}
%\def\be{\begin{equation*}}
%\def\ee{\end{equation*}}
%\def\bea{\begin{eqnarray*}}
%\def\eea{\end{eqnarray*}}
%\def\bi{\begin{itemize}}
%\def\ei{\end{itemize}}
%\def\fmn{\ensuremath{F_{\mu\nu}}}
%\def\fMN{\ensuremath{F^{\mu\nu}}}
%\def\bc{\begin{center}}
%\def\ec{\end{center}}
%\def\nus{$\nu$s}

\def\adagger{\ensuremath{a_{p\sigma}^\dagger}}
\def\lineacross{\noindent\rule{\textwidth}{1pt}}

\newcommand{\multiline}[1] {
\begin{tabular} {|l}
#1
\end{tabular}
}

\newcommand{\multilineNoLine}[1] {
\begin{tabular} {l}
#1
\end{tabular}
}



\newcommand{\lineTwo}[2] {
\begin{tabular} {|l}
#1 \\
#2
\end{tabular}
}

\newcommand{\rmt}[1] {
\textrm{#1}
}


%
% Units
%
\def\m{\ensuremath{\rmt{m}}}
\def\GeV{\ensuremath{\rmt{GeV}}}
\def\pt{\ensuremath{p_\rmt{T}}}


\def\parity{\ensuremath{\mathcal{P}}}

\usepackage{cancel}
\usepackage{ mathrsfs }
\def\bigL{\ensuremath{\mathscr{L}}}

\usepackage{ dsfont }

\def\nus{$\nu$s}
\def\nue{\ensuremath{\nu_e}}
\def\numu{\ensuremath{\nu_\mu}}
\def\nutau{\ensuremath{\nu_\tau}}
\def\nualpha{\ensuremath{\nu_\alpha}}
\def\nuone{\ensuremath{\nu_1}}
\def\nutwo{\ensuremath{\nu_2}}
\def\nuthree{\ensuremath{\nu_3}}


\usepackage{fancyhdr}
\fancyhf{}



\lhead{\Large 33-211} % \hfill Introduction to Particle Physics \hfill Spring 2022}
\chead{\Large Physics 3 : Modern Essentials} % \hfill Spring 2022}
\rhead{\Large Spring 2025} % \hfill Introduction to Particle Physics \hfill Spring 2022}
\begin{document}
\thispagestyle{fancy}





%\begin{tabular}{c}
%{\large 33-444 \hfill Intro To Particle \hfill Spring 2019\\}
%\hline 
%\end{tabular}

\begin{center}
{\huge \textbf{Exam \#2}}
\large

\end{center}

{\large


\textbf{1) Chased by a comet }\hfill \textit{(5 points)}\\
A comet is chasing a spaceship.
Let $\beta$, P and E be the speed, momentum, and energy of the comet as seen by the astronaut when it hits the spaceship. 
In what way would the increasing the spaceship's speed alter the astronaut's perceived values of $\beta$, P and E?

\begin{itemize}
\item[a)] $\beta$, P and E will all be constant not change at all.
\item[b)] $\beta$, P and E will all decrease.
\item[c)] $\beta$, P will get smaller, E will not change.
\item[d)] $\beta$, E will get smaller, P will not change.
\item[e)] P and E will get smaller, $\beta$ will not change.
\end{itemize}

\vspace{0.5in}

\textbf{2) Chased by a photon }\hfill \textit{(5 points)}\\
A photon is chasing a spaceship.
Let $\beta$, P and E be the speed, momentum, and energy of the photon as seen by the astronaut when it hits the spaceship. 
In what way would the increasing the spaceship's speed alter the astronaut's perceived values of $\beta$, P and E?

\begin{itemize}
\item[a)] $\beta$, P and E will all be constant not change at all.
\item[b)] $\beta$, P and E will all decrease.
\item[c)] $\beta$, P will get smaller, E will not change.
\item[d)] $\beta$, E will get smaller, P will not change.
\item[e)] P and E will get smaller, $\beta$ will not change.
\end{itemize}

\vspace{0.5in}

\textbf{3) Strong Box }\hfill \textit{(3 points)}\\
Suppose an atomic bomb was exploded in a box that was strong enough to contain all the energy released by the bomb.
After the explosion the box would weigh:

\begin{itemize}
\item[a)] more than before the explosion
\item[b)] less than before
\item[c)] the same as before
\end{itemize}


\clearpage

\textbf{4) General Relativity }\hfill \textit{(5 points)}\\
Which of the following statements are TRUE about general relativity
\begin{itemize}
%\item[a)] Clocks higher up in a gravitational potential run faster than clocks lower down.
\item[a)] General relativity is fundamentally a theory of gravity. 
\item[b)] Measurements made in a uniform gravitational field are indistinguishable from measurements made is a uniformly accelerating reference frame.
\item[c)] The bending of light in a gravitational field is an illusion: viewed from far away the light path is a straight line.
\item[d)] In general relativity, the internal and gravitational mass of an object are not exactly the same in a gravitational potential well.
\item[e)] General relativity implies a direct connection between space-time curvature and the electro-magnetic force
\end{itemize}

\vspace{0.5in}

\textbf{5) Mass }\hfill \textit{(12 points)}\\

Consider three particles A, B, C.
Particle A has 10 \GeV\ of total energy and is moving at $\beta = \frac{3}{5}$.
Particle B has 8 \GeV\ of total energy and 2 \GeV\ of momentum.
Particle C has 12 \GeV\ of total energy and 3 \GeV\ of kinetic energy.
Which particle is the most massive ?
Which is the least massive?


\vspace{0.1in}

\clearpage

\textbf{6) Elastic Collisions }\hfill \textit{(25 points)}\\
A particle of mass m in incident with kinetic energy $KE$ on an identical particle at rest relative to an inertial system S.
The collision is elastic and such that the outgoing scattered particles make the same angle wrt to the x-axis.  


\begin{itemize}
\item[a)]Find the angle between the direction of motion of the two particles according to non-relativistic (Newtonian) physics 

\vspace{3.in}

\item[b)]Find the angle between the direction of motion of the two particles (or the cosine of the angle) assuming relativistic physics. (Note: $\cos \theta = 2 \cos^2 \frac{\theta}{2} - 1$)

\vspace{3in}

\item[c)]What is the angle in the ultra-relativistic  ($KE >> m$) limit ? 

\end{itemize}



\clearpage



\textbf{7) Nuclear Physics }\hfill \textit{(15 points)}\\

An unstable nucleus A (with mass $M_1$ = 50 \GeV\ and proper lifetime = 5$s$) decays into a photon and another unstable nucleus B (with mass $M_2$ = 40 \GeV\ and proper lifetime = 1$s$).
What is the energy of the photon and the lifetime of nucleus B as observed from the rest frame of nucleus A?


\clearpage
\textbf{8) Testing the Standard Model }\hfill \textit{(20 points)}\\
The Standard Model of particle physics predicts a neutral particle, the Z boson, which has a mass of 100 GeV.
The Z can be produced in electron-positron collisions in the reaction $e^+ + e^- \rightarrow Z$.

\begin{itemize}
\item[a)]In a ``fixed target experiment'' the electron is at rest in the lab frame. 
How much kinetic energy in the lab frame does the positron need to in order to produce a Z ?

\vspace{3.75in}

\item[b)] In a ``collider'' the particles move with the same speed in opposite directions in the lab frame.
How much kinetic in the lab frame do the electron and positron need in order to produce a Z in a collider?
\end{itemize}


} % Begning Large
\end{document}
