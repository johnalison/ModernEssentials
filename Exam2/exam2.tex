\documentclass[paper=letter,11pt]{scrartcl}

\KOMAoptions{headinclude=true, footinclude=false}
\KOMAoptions{DIV=14, BCOR=5mm}
\KOMAoptions{numbers=noendperiod}
\KOMAoptions{parskip=half}
\addtokomafont{disposition}{\rmfamily}
\addtokomafont{part}{\LARGE}
\addtokomafont{descriptionlabel}{\rmfamily}
%\setkomafont{pageheadfoot}{\normalsize\sffamily}
\setkomafont{pagehead}{\normalsize\rmfamily}
%\setkomafont{publishers}{\normalsize\rmfamily}
\setkomafont{caption}{\normalfont\small}
\setcapindent{0pt}
\deffootnote[1em]{1em}{1em}{\textsuperscript{\thefootnotemark}\ }


\usepackage{amsmath}
\usepackage[varg]{txfonts}
\usepackage[T1]{fontenc}
\usepackage{graphicx}
\usepackage{xcolor}
\usepackage[american]{babel}
% hyperref is needed in many places, so include it here
\usepackage{hyperref}

\usepackage{xspace}
\usepackage{multirow}
\usepackage{float}


\usepackage{braket}
\usepackage{bbm}
\usepackage{relsize}
\usepackage{tcolorbox}


\def\x{\ensuremath{x}}
\def\xp{\ensuremath{x'}}
\def\t{\ensuremath{t}}
\def\tp{\ensuremath{t'}}
\def\v{\ensuremath{v}}
%\def\nus{$\nu$s}

%\def\ketY{\ensuremath{\ket {\Psi}}}
%\def\iGeV{\ensuremath{\textrm{GeV}^{-1}}}
%%\def\mp{\ensuremath{m_{\textrm{proton}}}}
%\def\rp{\ensuremath{r_{\textrm{proton}}}}
%\def\me{\ensuremath{m_{\textrm{electron}}}}
%\def\aG{\ensuremath{\alpha_G}}
%\def\rAtom{\ensuremath{r_{\textrm{atom}}}}
%\def\rNucl{\ensuremath{r_{\textrm{nucleus}}}}
%\def\GN{\ensuremath{\textrm{G}_\textrm{N}}}
%\def\ketX{\ensuremath{\ket{\vec{x}}}}
%\def\ve{\ensuremath{\vec{\epsilon}}}
%
%
%\def\ABCDMatrix{\ensuremath{\begin{pmatrix} A &  B  \\ C  & D \end{pmatrix}}}
%\def\xyprime{\ensuremath{\begin{pmatrix} x' \\ y' \end{pmatrix}}}
%\def\xyprimeT{\ensuremath{\begin{pmatrix} x' &  y' \end{pmatrix}}}
%\def\xy{\ensuremath{\begin{pmatrix} x \\ y \end{pmatrix}}}
%\def\xyT{\ensuremath{\begin{pmatrix} x & y \end{pmatrix}}}
%
%\def\IMatrix{\ensuremath{\begin{pmatrix} 0 &  1  \\ -1  & 0 \end{pmatrix}}}
%\def\IBoostMatrix{\ensuremath{\begin{pmatrix} 0 &  1  \\ 1  & 0 \end{pmatrix}}}
%\def\JThree{\ensuremath{\begin{pmatrix}    0 & -i & 0  \\ i & 0  & 0 \\ 0 & 0 & 0 \end{pmatrix}}} 
%\def\JTwo{\ensuremath{\begin{bmatrix}    0 & 0 & -i  \\ 0 & 0  & 0 \\ i & 0 & 0 \end{bmatrix}}}
%\def\JOne{\ensuremath{\begin{bmatrix}    0 & 0 & 0  \\ 0 & 0  & -i \\ 0 & i & 0 \end{bmatrix}}}
%\def\etamn{\ensuremath{\eta_{\mu\nu}}}
%\def\Lmn{\ensuremath{\Lambda^\mu_\nu}}
%\def\dmn{\ensuremath{\delta^\mu_\nu}}
%\def\wmn{\ensuremath{\omega^\mu_\nu}}
%\def\be{\begin{equation*}}
%\def\ee{\end{equation*}}
%\def\bea{\begin{eqnarray*}}
%\def\eea{\end{eqnarray*}}
%\def\bi{\begin{itemize}}
%\def\ei{\end{itemize}}
%\def\fmn{\ensuremath{F_{\mu\nu}}}
%\def\fMN{\ensuremath{F^{\mu\nu}}}
%\def\bc{\begin{center}}
%\def\ec{\end{center}}
%\def\nus{$\nu$s}

\def\adagger{\ensuremath{a_{p\sigma}^\dagger}}
\def\lineacross{\noindent\rule{\textwidth}{1pt}}

\newcommand{\multiline}[1] {
\begin{tabular} {|l}
#1
\end{tabular}
}

\newcommand{\multilineNoLine}[1] {
\begin{tabular} {l}
#1
\end{tabular}
}



\newcommand{\lineTwo}[2] {
\begin{tabular} {|l}
#1 \\
#2
\end{tabular}
}

\newcommand{\rmt}[1] {
\textrm{#1}
}


%
% Units
%
\def\m{\ensuremath{\rmt{m}}}
\def\GeV{\ensuremath{\rmt{GeV}}}
\def\pt{\ensuremath{p_\rmt{T}}}


\def\parity{\ensuremath{\mathcal{P}}}

\usepackage{cancel}
\usepackage{ mathrsfs }
\def\bigL{\ensuremath{\mathscr{L}}}

\usepackage{ dsfont }

\def\nus{$\nu$s}
\def\nue{\ensuremath{\nu_e}}
\def\numu{\ensuremath{\nu_\mu}}
\def\nutau{\ensuremath{\nu_\tau}}
\def\nualpha{\ensuremath{\nu_\alpha}}
\def\nuone{\ensuremath{\nu_1}}
\def\nutwo{\ensuremath{\nu_2}}
\def\nuthree{\ensuremath{\nu_3}}


\usepackage{fancyhdr}
\fancyhf{}



\lhead{\Large 33-211} % \hfill Introduction to Particle Physics \hfill Spring 2022}
\chead{\Large Physics 3 : Modern Essentials} % \hfill Spring 2022}
\rhead{\Large Spring 2023} % \hfill Introduction to Particle Physics \hfill Spring 2022}
\begin{document}
\thispagestyle{fancy}





%\begin{tabular}{c}
%{\large 33-444 \hfill Intro To Particle \hfill Spring 2019\\}
%\hline 
%\end{tabular}

\begin{center}
{\huge \textbf{Exam \#1}}
\large

\end{center}

{\large


\textbf{1) Chased by a comet }\hfill \textit{(4 points)}\\


\begin{itemize}
\item[a)] 
\item[b)] 
\item[c)] 
\item[d)] 
\item[e)] 
\end{itemize}

\vspace{0.1in}

\textbf{2) Chased by a photon }\hfill \textit{(4 points)}\\

\begin{itemize}
\item[a)] 
\item[b)] 
\item[c)] 
\item[d)] 
\end{itemize}

\vspace{0.1in}

\textbf{2) Strong Box }\hfill \textit{(4 points)}\\

\begin{itemize}
\item[a)] 
\item[b)] 
\item[c)] 
\item[d)] 
\end{itemize}

\vspace{0.1in}


\textbf{3) General Relativity }\hfill \textit{(8 points)}\\
Are true about general relativity
\begin{itemize}
\item[a)] Clocks higher up in a gravitaional potential run faster than clocks lower down.
\item[c)] Measurements made in a uniform gravitaional feild are indistinguishable from measurements made is a uniformly accelerating reference frame.
\item[b)] The bending of light in a gravitational field is an illusion: viewed from far away the light path is a straight line.
\item[c)] In GR, the interial and gravitational mass of an object are not exactly the same in a gravitational potential well.
\item[d)] 
\item[e)] 
\item[f)] 
\item[g)] 
\end{itemize}

\clearpage

\textbf{X) Mass }\hfill \textit{(Y points)}\\
Consider three particles A, B, C.
Particle A has 10 \GeV\ of total energy and is moving at $\beta = \frac{3}{5}$.
Particle B has 8 \GeV\ of total energy and 2 \GeV\ of momentum.
Particle C has 12 \GeV\ of total energy and 3 \GeV\ of kinetic energy.
Which particle is the most massive ?
Which is the least massive?


\vspace{0.1in}

% Collider vs fixed target experment.
%  Testing a theory with new neutral particle Z'
%  Collider vs Fixed target


%
%  Show angle of the symmetric collision
%

\textbf{X) Nuclear Physics }\hfill \textit{(Y points)}\\

An unstable nucleus A (with mass $M_1$ = 50 \GeV\ and proper lifetime = 5$s$) decays into a photon and another unstable nucleus B (with mass $M_2 = 40\GeV$ and proper lifetime = $1s$).
What is the energy of the photon and the lifetime of nucleus B as observed from the rest frame of nucleus A?


} % Begning Large
\end{document}
