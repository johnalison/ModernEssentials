\documentclass[paper=letter,11pt]{scrartcl}

\KOMAoptions{headinclude=true, footinclude=false}
\KOMAoptions{DIV=14, BCOR=5mm}
\KOMAoptions{numbers=noendperiod}
\KOMAoptions{parskip=half}
\addtokomafont{disposition}{\rmfamily}
\addtokomafont{part}{\LARGE}
\addtokomafont{descriptionlabel}{\rmfamily}
%\setkomafont{pageheadfoot}{\normalsize\sffamily}
\setkomafont{pagehead}{\normalsize\rmfamily}
%\setkomafont{publishers}{\normalsize\rmfamily}
\setkomafont{caption}{\normalfont\small}
\setcapindent{0pt}
\deffootnote[1em]{1em}{1em}{\textsuperscript{\thefootnotemark}\ }


\usepackage{amsmath}
\usepackage[varg]{txfonts}
\usepackage[T1]{fontenc}
\usepackage{graphicx}
\usepackage{xcolor}
\usepackage[american]{babel}
% hyperref is needed in many places, so include it here
\usepackage{hyperref}

\usepackage{xspace}
\usepackage{multirow}
\usepackage{float}


\usepackage{braket}
\usepackage{bbm}
\usepackage{relsize}
\usepackage{tcolorbox}


\def\x{\ensuremath{x}}
\def\xp{\ensuremath{x'}}
\def\t{\ensuremath{t}}
\def\tp{\ensuremath{t'}}
\def\v{\ensuremath{v}}
%\def\nus{$\nu$s}

%\def\ketY{\ensuremath{\ket {\Psi}}}
%\def\iGeV{\ensuremath{\textrm{GeV}^{-1}}}
%%\def\mp{\ensuremath{m_{\textrm{proton}}}}
%\def\rp{\ensuremath{r_{\textrm{proton}}}}
%\def\me{\ensuremath{m_{\textrm{electron}}}}
%\def\aG{\ensuremath{\alpha_G}}
%\def\rAtom{\ensuremath{r_{\textrm{atom}}}}
%\def\rNucl{\ensuremath{r_{\textrm{nucleus}}}}
%\def\GN{\ensuremath{\textrm{G}_\textrm{N}}}
%\def\ketX{\ensuremath{\ket{\vec{x}}}}
%\def\ve{\ensuremath{\vec{\epsilon}}}
%
%
%\def\ABCDMatrix{\ensuremath{\begin{pmatrix} A &  B  \\ C  & D \end{pmatrix}}}
%\def\xyprime{\ensuremath{\begin{pmatrix} x' \\ y' \end{pmatrix}}}
%\def\xyprimeT{\ensuremath{\begin{pmatrix} x' &  y' \end{pmatrix}}}
%\def\xy{\ensuremath{\begin{pmatrix} x \\ y \end{pmatrix}}}
%\def\xyT{\ensuremath{\begin{pmatrix} x & y \end{pmatrix}}}
%
%\def\IMatrix{\ensuremath{\begin{pmatrix} 0 &  1  \\ -1  & 0 \end{pmatrix}}}
%\def\IBoostMatrix{\ensuremath{\begin{pmatrix} 0 &  1  \\ 1  & 0 \end{pmatrix}}}
%\def\JThree{\ensuremath{\begin{pmatrix}    0 & -i & 0  \\ i & 0  & 0 \\ 0 & 0 & 0 \end{pmatrix}}} 
%\def\JTwo{\ensuremath{\begin{bmatrix}    0 & 0 & -i  \\ 0 & 0  & 0 \\ i & 0 & 0 \end{bmatrix}}}
%\def\JOne{\ensuremath{\begin{bmatrix}    0 & 0 & 0  \\ 0 & 0  & -i \\ 0 & i & 0 \end{bmatrix}}}
%\def\etamn{\ensuremath{\eta_{\mu\nu}}}
%\def\Lmn{\ensuremath{\Lambda^\mu_\nu}}
%\def\dmn{\ensuremath{\delta^\mu_\nu}}
%\def\wmn{\ensuremath{\omega^\mu_\nu}}
%\def\be{\begin{equation*}}
%\def\ee{\end{equation*}}
%\def\bea{\begin{eqnarray*}}
%\def\eea{\end{eqnarray*}}
%\def\bi{\begin{itemize}}
%\def\ei{\end{itemize}}
%\def\fmn{\ensuremath{F_{\mu\nu}}}
%\def\fMN{\ensuremath{F^{\mu\nu}}}
%\def\bc{\begin{center}}
%\def\ec{\end{center}}
%\def\nus{$\nu$s}

\def\adagger{\ensuremath{a_{p\sigma}^\dagger}}
\def\lineacross{\noindent\rule{\textwidth}{1pt}}

\newcommand{\multiline}[1] {
\begin{tabular} {|l}
#1
\end{tabular}
}

\newcommand{\multilineNoLine}[1] {
\begin{tabular} {l}
#1
\end{tabular}
}



\newcommand{\lineTwo}[2] {
\begin{tabular} {|l}
#1 \\
#2
\end{tabular}
}

\newcommand{\rmt}[1] {
\textrm{#1}
}


%
% Units
%
\def\m{\ensuremath{\rmt{m}}}
\def\GeV{\ensuremath{\rmt{GeV}}}
\def\pt{\ensuremath{p_\rmt{T}}}


\def\parity{\ensuremath{\mathcal{P}}}

\usepackage{cancel}
\usepackage{ mathrsfs }
\def\bigL{\ensuremath{\mathscr{L}}}

\usepackage{ dsfont }

\def\nus{$\nu$s}
\def\nue{\ensuremath{\nu_e}}
\def\numu{\ensuremath{\nu_\mu}}
\def\nutau{\ensuremath{\nu_\tau}}
\def\nualpha{\ensuremath{\nu_\alpha}}
\def\nuone{\ensuremath{\nu_1}}
\def\nutwo{\ensuremath{\nu_2}}
\def\nuthree{\ensuremath{\nu_3}}


\usepackage{fancyhdr}
\fancyhf{}



\lhead{\Large 33-211} % \hfill Introduction to Particle Physics \hfill Spring 2022}
\chead{\Large Physics 3 : Modern Essentials} % \hfill Spring 2022}
\rhead{\Large Spring 2023} % \hfill Introduction to Particle Physics \hfill Spring 2022}
\begin{document}
\thispagestyle{fancy}


\begin{center}
{\huge \textbf{Exam \#3}}
\large

\end{center}

{\large



\textbf{1) Compton Scattering. }\hfill \textit{X Points}\\
In Compton scattering of X-rays from electrons the outgoing X-rays have shorter wavelength than the incoming X-rays: True or False.

\vspace*{0.4in}

\textbf{2) Bug fixing the Periodic Table. }\hfill \textit{Y Points}\\
Before Mosely, the periodic table was ordered by atomic mass. Mosley showed that it should be ordered by atomic charge.  Describe how was he able to measure the charge on the nucleas Z.

\vfill

\textbf{3) The Frank-Hertz Experiment. }\hfill \textit{X Points}\\
Describe how the quantization of atomic energies can be observed by looking at electron collisions in gas.

\vfill

\clearpage

\textbf{4) Cathode rays}\hfill \textit{X Points}\\
Describe Tomposons Cathode-ray experiment that disovered the electron.
What did property of electrons did he measure ?
What was surprising about his result ?

\vspace*{2.4in}


\textbf{5) Black body radiation}\hfill \textit{X Points}\\
Sketch a comparison of the observed radiation distrbution (power radiated / unit area) vs wavelength for a blackbody with temperature 6000 K (peak at 800 nm.)
Overlay a sketch of the classical prediction.


\vspace*{3in}

In natural units, T has units \GeV.  What are the units of power/area in \GeV ? How does the total power output by a star (which is a blackbody) scale with T ?

\vfill

\clearpage

\begin{minipage}{\textwidth}
\textbf{6) Development of the Atomic model. }\hfill \textit{6 Points}
\begin{itemize}
\item[-]Describe Rutherford's experiment testing the atomic model.
\vspace*{1.in}
\item[-]What was unexpected about his results ?
\vspace*{1.in}
\item[-]How did this change our picture of the atom?
\end{itemize}
\end{minipage}

\begin{minipage}{\textwidth}
\textbf{7) The Bohr Model of an Atom. }\hfill \textit{4 Points}
\begin{itemize}
\item[-] How was Bohr's model different from the planetary model of atoms that came before it ?
\vspace*{1in}
\item[-] What problems did it address ?
\vspace*{1in}
If a model of a Bohr hyrdrogen atom had a proton the size of an orange (5 cm), how far away would the first electron orbit be?
\end{itemize}
\end{minipage}

\vspace{0.25in}

\begin{minipage}{\textwidth}
\textbf{8) Particles As Waves } \hfill \textit{10 Points}
\begin{itemize}
\item[-] Use the uncertainty principle to estimate the energy of an electron in an infinite square well of width L.
\vspace*{2in}
\item[-] Use the uncertainty principle to estimate the size of a hydrogen atom (V = -$\alpha/r$). \\ \textit{(Express your result in terms of $\hbar$, m. $\alpha$ )}
\vspace*{1in}
\end{itemize}
\end{minipage}
\vspace{0.25in}


\clearpage

\textbf{9) Photoelectric Effect }\hfill \textit{(X points)}\\

Soppose we are shinning a beam of ultraviolet light of known wavelength onto a metal surface with a binding energy of electrons (or work function) is 2.0 eV

\begin{itemize}
\item[a)] If this light liberates photoelectrons of maximum kinetic energy 3.0 eV, what is the wavelength of the light?
\vfill
\item[b)] Suppose the intesity of the light impinging upon the metal surface is tripled. What do you expect to happen to the number of electrons liberated from the surface? Explain.
\vfill
\item[c)] Suppose instead that the wavelength of the light is tripled. What do you expect to happen to the number of electrons liberated from the surface? Explain.
\vfill
\end{itemize}


\clearpage

\textbf{10) Positronium }\hfill \textit{(X points)}\\

\begin{itemize}
\item[a)] Estimate the size and binding energy of a hydrogen atom using the uncertainty principle. 
\vfill
\item[b)] Estimate the size and binding energy of positronium using the uncertainty principle. Positronium is the bound state of an electron and an anti-electron (or positron).
\vfill
\end{itemize}



} % Begning Large
\end{document}
