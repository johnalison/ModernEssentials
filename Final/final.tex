\documentclass[paper=letter,11pt]{scrartcl}

\KOMAoptions{headinclude=true, footinclude=false}
\KOMAoptions{DIV=14, BCOR=5mm}
\KOMAoptions{numbers=noendperiod}
\KOMAoptions{parskip=half}
\addtokomafont{disposition}{\rmfamily}
\addtokomafont{part}{\LARGE}
\addtokomafont{descriptionlabel}{\rmfamily}
%\setkomafont{pageheadfoot}{\normalsize\sffamily}
\setkomafont{pagehead}{\normalsize\rmfamily}
%\setkomafont{publishers}{\normalsize\rmfamily}
\setkomafont{caption}{\normalfont\small}
\setcapindent{0pt}
\deffootnote[1em]{1em}{1em}{\textsuperscript{\thefootnotemark}\ }


\usepackage{amsmath}
\usepackage[varg]{txfonts}
\usepackage[T1]{fontenc}
\usepackage{graphicx}
\usepackage{xcolor}
\usepackage[american]{babel}
% hyperref is needed in many places, so include it here
\usepackage{hyperref}

\usepackage{xspace}
\usepackage{multirow}
\usepackage{float}


\usepackage{braket}
\usepackage{bbm}
\usepackage{relsize}
\usepackage{tcolorbox}


\def\x{\ensuremath{x}}
\def\xp{\ensuremath{x'}}
\def\t{\ensuremath{t}}
\def\tp{\ensuremath{t'}}
\def\v{\ensuremath{v}}
%\def\nus{$\nu$s}

%\def\ketY{\ensuremath{\ket {\Psi}}}
%\def\iGeV{\ensuremath{\textrm{GeV}^{-1}}}
%%\def\mp{\ensuremath{m_{\textrm{proton}}}}
%\def\rp{\ensuremath{r_{\textrm{proton}}}}
%\def\me{\ensuremath{m_{\textrm{electron}}}}
%\def\aG{\ensuremath{\alpha_G}}
%\def\rAtom{\ensuremath{r_{\textrm{atom}}}}
%\def\rNucl{\ensuremath{r_{\textrm{nucleus}}}}
%\def\GN{\ensuremath{\textrm{G}_\textrm{N}}}
%\def\ketX{\ensuremath{\ket{\vec{x}}}}
%\def\ve{\ensuremath{\vec{\epsilon}}}
%
%
%\def\ABCDMatrix{\ensuremath{\begin{pmatrix} A &  B  \\ C  & D \end{pmatrix}}}
%\def\xyprime{\ensuremath{\begin{pmatrix} x' \\ y' \end{pmatrix}}}
%\def\xyprimeT{\ensuremath{\begin{pmatrix} x' &  y' \end{pmatrix}}}
%\def\xy{\ensuremath{\begin{pmatrix} x \\ y \end{pmatrix}}}
%\def\xyT{\ensuremath{\begin{pmatrix} x & y \end{pmatrix}}}
%
%\def\IMatrix{\ensuremath{\begin{pmatrix} 0 &  1  \\ -1  & 0 \end{pmatrix}}}
%\def\IBoostMatrix{\ensuremath{\begin{pmatrix} 0 &  1  \\ 1  & 0 \end{pmatrix}}}
%\def\JThree{\ensuremath{\begin{pmatrix}    0 & -i & 0  \\ i & 0  & 0 \\ 0 & 0 & 0 \end{pmatrix}}} 
%\def\JTwo{\ensuremath{\begin{bmatrix}    0 & 0 & -i  \\ 0 & 0  & 0 \\ i & 0 & 0 \end{bmatrix}}}
%\def\JOne{\ensuremath{\begin{bmatrix}    0 & 0 & 0  \\ 0 & 0  & -i \\ 0 & i & 0 \end{bmatrix}}}
%\def\etamn{\ensuremath{\eta_{\mu\nu}}}
%\def\Lmn{\ensuremath{\Lambda^\mu_\nu}}
%\def\dmn{\ensuremath{\delta^\mu_\nu}}
%\def\wmn{\ensuremath{\omega^\mu_\nu}}
%\def\be{\begin{equation*}}
%\def\ee{\end{equation*}}
%\def\bea{\begin{eqnarray*}}
%\def\eea{\end{eqnarray*}}
%\def\bi{\begin{itemize}}
%\def\ei{\end{itemize}}
%\def\fmn{\ensuremath{F_{\mu\nu}}}
%\def\fMN{\ensuremath{F^{\mu\nu}}}
%\def\bc{\begin{center}}
%\def\ec{\end{center}}
%\def\nus{$\nu$s}

\def\adagger{\ensuremath{a_{p\sigma}^\dagger}}
\def\lineacross{\noindent\rule{\textwidth}{1pt}}

\newcommand{\multiline}[1] {
\begin{tabular} {|l}
#1
\end{tabular}
}

\newcommand{\multilineNoLine}[1] {
\begin{tabular} {l}
#1
\end{tabular}
}



\newcommand{\lineTwo}[2] {
\begin{tabular} {|l}
#1 \\
#2
\end{tabular}
}

\newcommand{\rmt}[1] {
\textrm{#1}
}


%
% Units
%
\def\m{\ensuremath{\rmt{m}}}
\def\GeV{\ensuremath{\rmt{GeV}}}
\def\pt{\ensuremath{p_\rmt{T}}}


\def\parity{\ensuremath{\mathcal{P}}}

\usepackage{cancel}
\usepackage{ mathrsfs }
\def\bigL{\ensuremath{\mathscr{L}}}

\usepackage{ dsfont }

\def\nus{$\nu$s}
\def\nue{\ensuremath{\nu_e}}
\def\numu{\ensuremath{\nu_\mu}}
\def\nutau{\ensuremath{\nu_\tau}}
\def\nualpha{\ensuremath{\nu_\alpha}}
\def\nuone{\ensuremath{\nu_1}}
\def\nutwo{\ensuremath{\nu_2}}
\def\nuthree{\ensuremath{\nu_3}}


\usepackage{fancyhdr}
\fancyhf{}



\lhead{\Large 33-211} % \hfill Introduction to Particle Physics \hfill Spring 2022}
\chead{\Large Physics 3 : Modern Essentials} % \hfill Spring 2022}
\rhead{\Large Spring 2023} % \hfill Introduction to Particle Physics \hfill Spring 2022}
\begin{document}
\thispagestyle{fancy}





%\begin{tabular}{c}
%{\large 33-444 \hfill Intro To Particle \hfill Spring 2019\\}
%\hline 
%\end{tabular}

\begin{center}
{\huge \textbf{Exam \#1}}
\large

\end{center}

{\large



\textbf{1) Chased by a comet }\hfill \textit{(5 points)}\\
A comet is chasing a spaceship.
Let $\beta$, P and E be the speed, momentum, and energy of the comet as seen by the astronaut when it hits the spaceship. 
In what way would the increasing the spaceship's speed alter the astronaut's perceived values of $\beta$, P and E?

\begin{itemize}
\item[a)] $\beta$, P and E will all be constant not change at all.
\item[b)] $\beta$, P and E will all decrease.
\item[c)] $\beta$, P will get smaller, E will not change.
\item[d)] $\beta$, E will get smaller, P will not change.
\item[e)] P and E will get smaller, $\beta$ will not change.
\end{itemize}

\vspace{0.5in}

\textbf{2) Chased by a photon }\hfill \textit{(5 points)}\\
A photon is chasing a spaceship.
Let $\beta$, P and E be the speed, momentum, and energy of the photon as seen by the astronaut when it hits the spaceship. 
In what way would the increasing the spaceship's speed alter the astronaut's perceived values of $\beta$, P and E?

\begin{itemize}
\item[a)] $\beta$, P and E will all be constant not change at all.
\item[b)] $\beta$, P and E will all decrease.
\item[c)] $\beta$, P will get smaller, E will not change.
\item[d)] $\beta$, E will get smaller, P will not change.
\item[e)] P and E will get smaller, $\beta$ will not change.
\end{itemize}

\vspace{0.5in}


\textbf{3) Invariants }\hfill \textit{(8 points)}\\
Which of the following are invariant (ie: agreed on by all inertial observers)?
\begin{itemize}
\item[a)] time ordering of time-like separated events
\item[c)] component of the velocity of a projectile parallel to relative direction of motion
\item[b)] component of the velocity of a projectile perpendicular to relative direction of motion
\item[c)] time between events
\item[d)] distance between events
\item[e)] total particle speed when beta < 1
\item[f)] total particle speed when beta = 1
\item[g)] proper time along a world line
\end{itemize}

\clearpage

\textbf{4) Relative velocities }\hfill \textit{(12 points)}\\

A rocket ship moving at 0.5c wrt earth fires a missile which moves at 0.8c wrt the rocket.
What is the speed of the rocket wrt earth, assuming classical physics (Galilean transformations)?
What is the speed of the rocket wrt earth, assuming relativity (Lorentz transformations)?

\vspace{2.8in}

\textbf{5) Causality}  \hfill \textit{(12 points)}\\
You are located at the origin of the S frame: (x,t)=(0,0).
Your friend is located at the origin of the S` which is moving to the right at $\beta = 0.99$ wrt the S frame in the usual way with origins coinciding at (0,0).
Consider the following space-time events (coordinates in S):\\
\begin{tabular}{clr}
  & (x,& t)\\
  \hline
  A = & (0   , & 2)\\
  B = & (2.01, & 2)\\
  C = & (1.99, & 2)\\
  D = & (2,    & 0)\\
  E = & (0,    & -2)\\
  F = & (2,    & -1.99)\\
  G = & (2,    & -2.01)\\
\end{tabular}

\vspace{0.1in}

a) Which events can you causally effect ?\\

b) Which events can your friend causally effect ?\\

c) Which events can casually effect you?\\

d) Which events can casually effected your friend?

%a) Argue with a space diagram that casualty (ie: time-ordering) is preserved 
%is causes propogate at beta < 0. 
%
%\vspace{2in}
%b) Argue with a space diagram that casality (ie: time-ordering) is NOT preserved 
%if beta > 1
%\vspace{2in}
\clearpage

\textbf{6) Olsen twins}  \hfill \textit{(15 points)}\\
Mary-Kate and Ashley are famous child twin actors.
They decide to try to prolong their combined effective career by sending one of them on a high-speed round-trip journey. 
Mary-Kay travels at speed $\beta = 24/25$ away from earth for 7 years as measured by her.
She then turns around and returns to earth with speed $\beta = 24/25$.

a) How much older is Mary-Kate when she returns ?

b) How much older is Ashley when Mary-Kate returns ?


\clearpage

\textbf{7) Analyze the Michelson and Morely experiment in \underline{the fixed star frame}.}  \hfill \textit{(20 points)}\\
Show that the null effect  ($\Delta t = 0$) can be accounted for with Relativity.



\textbf{7) Nuclear Physics }\hfill \textit{(15 points)}\\
FLIP AROUND SO KNOW ?
An unstable nucleus A (with mass $M_1$ = 50 \GeV\ and proper lifetime = 5$s$) decays into a photon and another unstable nucleus B (with mass $M_2$ = 40 \GeV\ and proper lifetime = 1$s$).
What is the energy of the photon and the lifetime of nucleus B as observed from the rest frame of nucleus A?

\textbf{8) Photoelectric Effect }\hfill \textit{(10 points)}\\
Sketch graph stopping voltave vs

\textbf{9) Particles As Waves } \hfill \textit{15 Points}
\begin{itemize}
\item[a)] Use the uncertainty principle to estimate the energy of an electron in an infinite square well of width L.
\vspace*{2.5in}
\item[b)] Use the uncertainty principle to estimate the size of a hydrogen atom (V = -$\alpha/r$). \\ \textit{(Express your result in terms of $m_e$. $\alpha$ )}
          Do you need to worry about relativistic effects when analyzing the Hydrogen atom ?  Justify your answer.
%\vspace*{2.in}
\end{itemize}



} % Begning Large
\end{document}
