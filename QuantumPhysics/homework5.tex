\documentclass[paper=letter,11pt]{scrartcl}

\KOMAoptions{headinclude=true, footinclude=false}
\KOMAoptions{DIV=14, BCOR=5mm}
\KOMAoptions{numbers=noendperiod}
\KOMAoptions{parskip=half}
\addtokomafont{disposition}{\rmfamily}
\addtokomafont{part}{\LARGE}
\addtokomafont{descriptionlabel}{\rmfamily}
%\setkomafont{pageheadfoot}{\normalsize\sffamily}
\setkomafont{pagehead}{\normalsize\rmfamily}
%\setkomafont{publishers}{\normalsize\rmfamily}
\setkomafont{caption}{\normalfont\small}
\setcapindent{0pt}
\deffootnote[1em]{1em}{1em}{\textsuperscript{\thefootnotemark}\ }


\usepackage{amsmath}
\usepackage[varg]{txfonts}
\usepackage[T1]{fontenc}
\usepackage{graphicx}
\usepackage{xcolor}
\usepackage[american]{babel}
% hyperref is needed in many places, so include it here
\usepackage{hyperref}

\usepackage{xspace}
\usepackage{multirow}
\usepackage{float}


\usepackage{braket}
\usepackage{bbm}
\usepackage{relsize}
\usepackage{tcolorbox}


\def\x{\ensuremath{x}}
\def\xp{\ensuremath{x'}}
\def\t{\ensuremath{t}}
\def\tp{\ensuremath{t'}}
\def\v{\ensuremath{v}}
%\def\nus{$\nu$s}

%\def\ketY{\ensuremath{\ket {\Psi}}}
%\def\iGeV{\ensuremath{\textrm{GeV}^{-1}}}
%%\def\mp{\ensuremath{m_{\textrm{proton}}}}
%\def\rp{\ensuremath{r_{\textrm{proton}}}}
%\def\me{\ensuremath{m_{\textrm{electron}}}}
%\def\aG{\ensuremath{\alpha_G}}
%\def\rAtom{\ensuremath{r_{\textrm{atom}}}}
%\def\rNucl{\ensuremath{r_{\textrm{nucleus}}}}
%\def\GN{\ensuremath{\textrm{G}_\textrm{N}}}
%\def\ketX{\ensuremath{\ket{\vec{x}}}}
%\def\ve{\ensuremath{\vec{\epsilon}}}
%
%
%\def\ABCDMatrix{\ensuremath{\begin{pmatrix} A &  B  \\ C  & D \end{pmatrix}}}
%\def\xyprime{\ensuremath{\begin{pmatrix} x' \\ y' \end{pmatrix}}}
%\def\xyprimeT{\ensuremath{\begin{pmatrix} x' &  y' \end{pmatrix}}}
%\def\xy{\ensuremath{\begin{pmatrix} x \\ y \end{pmatrix}}}
%\def\xyT{\ensuremath{\begin{pmatrix} x & y \end{pmatrix}}}
%
%\def\IMatrix{\ensuremath{\begin{pmatrix} 0 &  1  \\ -1  & 0 \end{pmatrix}}}
%\def\IBoostMatrix{\ensuremath{\begin{pmatrix} 0 &  1  \\ 1  & 0 \end{pmatrix}}}
%\def\JThree{\ensuremath{\begin{pmatrix}    0 & -i & 0  \\ i & 0  & 0 \\ 0 & 0 & 0 \end{pmatrix}}} 
%\def\JTwo{\ensuremath{\begin{bmatrix}    0 & 0 & -i  \\ 0 & 0  & 0 \\ i & 0 & 0 \end{bmatrix}}}
%\def\JOne{\ensuremath{\begin{bmatrix}    0 & 0 & 0  \\ 0 & 0  & -i \\ 0 & i & 0 \end{bmatrix}}}
%\def\etamn{\ensuremath{\eta_{\mu\nu}}}
%\def\Lmn{\ensuremath{\Lambda^\mu_\nu}}
%\def\dmn{\ensuremath{\delta^\mu_\nu}}
%\def\wmn{\ensuremath{\omega^\mu_\nu}}
%\def\be{\begin{equation*}}
%\def\ee{\end{equation*}}
%\def\bea{\begin{eqnarray*}}
%\def\eea{\end{eqnarray*}}
%\def\bi{\begin{itemize}}
%\def\ei{\end{itemize}}
%\def\fmn{\ensuremath{F_{\mu\nu}}}
%\def\fMN{\ensuremath{F^{\mu\nu}}}
%\def\bc{\begin{center}}
%\def\ec{\end{center}}
%\def\nus{$\nu$s}

\def\adagger{\ensuremath{a_{p\sigma}^\dagger}}
\def\lineacross{\noindent\rule{\textwidth}{1pt}}

\newcommand{\multiline}[1] {
\begin{tabular} {|l}
#1
\end{tabular}
}

\newcommand{\multilineNoLine}[1] {
\begin{tabular} {l}
#1
\end{tabular}
}



\newcommand{\lineTwo}[2] {
\begin{tabular} {|l}
#1 \\
#2
\end{tabular}
}

\newcommand{\rmt}[1] {
\textrm{#1}
}


%
% Units
%
\def\m{\ensuremath{\rmt{m}}}
\def\GeV{\ensuremath{\rmt{GeV}}}
\def\pt{\ensuremath{p_\rmt{T}}}


\def\parity{\ensuremath{\mathcal{P}}}

\usepackage{cancel}
\usepackage{ mathrsfs }
\def\bigL{\ensuremath{\mathscr{L}}}

\usepackage{ dsfont }

\def\nus{$\nu$s}
\def\nue{\ensuremath{\nu_e}}
\def\numu{\ensuremath{\nu_\mu}}
\def\nutau{\ensuremath{\nu_\tau}}
\def\nualpha{\ensuremath{\nu_\alpha}}
\def\nuone{\ensuremath{\nu_1}}
\def\nutwo{\ensuremath{\nu_2}}
\def\nuthree{\ensuremath{\nu_3}}


\usepackage{fancyhdr}
\fancyhf{}


\def\xyprime{\ensuremath{\begin{pmatrix} x' \\ y' \end{pmatrix}}}


\lhead{\Large 33-211} % \hfill Introduction to Particle Physics \hfill Spring 2022}
\chead{\Large Physics 3 : Modern Essentials} % \hfill Spring 2022}
\rhead{\Large Spring 2025} % \hfill Introduction to Particle Physics \hfill Spring 2022}
\begin{document}
\thispagestyle{fancy}





%\begin{tabular}{c}
%{\large 33-444 \hfill Intro To Particle \hfill Spring 2022\\}
%\hline 
%\end{tabular}

\begin{center}
{\huge \textbf{Homework Set \#5}}
\large

{\textbf{ Due Date:} Before class Friday March 14th  }
\end{center}

\textbf{1) Reading } \hfill \textit{(2 points)}\\
Read chapter 3.

\vspace*{0.25in}


{\large

\textbf{2) Discovery of the electron } \hfill  \textit{(10 points)}\\
The following table gives the Electronic and Magnetic fields and displacements measured by Thompson in his experiment on cathode rays, all in standard SI mks units.
The E\&M fields where applied at right angles to the cathode ray motion over a region of 0.05 m.
The rays where then allowed to travel freely for 1.1 m before striking end of the tube where the displacements were measured.
Thomson adjusted the magnetic field to give the same deflection 

\begin{table}[h]
\centering
\begin{tabular}{cccc}
Electric Field &  Electric deflection & Magnetic field & Magnetic deflection \\
(N/C) & (m) &  (N / A m)  &  (m) \\
\hline
$1.5 \times 10^4$ & 0.08 & $5.5\times 10^{-4}$ & 0.08 \\
$1.8 \times 10^4$ & 0.06 & $5.0\times 10^{-4}$ & 0.06 \\
\end{tabular}
\end{table}

\begin{itemize}
\item[-]Calculate the ray's q/m for each series of measurements.
\item[-]Calculate the ray's v for each series of measurements. Is it OK to use Newtonian physics or should relativistic corrections be considered ?
\end{itemize}

\vspace*{0.25in}

\textbf{3) Black body radiation } \hfill \textit{(30 points)}
\begin{itemize}
\item[a)] Find and enumerate the different possible standing waves on a 1D string of length L with the condition that the amplitude goes to zero at the boundaries.
         What are the allowed values of $k=\frac{2\pi}{\lambda}$ ?
\item[b)] Find and enumerate the different possible standing waves in a 3D box with sides of length L with the condition that the amplitude goes to zero at the boundaries.
         What are the allowed values of $k_x,k_y$,and $k_z$ ?
\item[c)] Show that number of modes for a given $k = |\vec{k}|$ is $L^3 \frac{k^2}{\pi^2} dk$.
\item[d)] In the previous step you derived $g_k(k)$, the number of modes per k per volume, find $g_\lambda(\lambda)$ from the constraint $g_\lambda(\lambda)d\lambda = g_k(k)dk$.
\item[e)] Derive Rayleigh-Jeans equation (ie: the classical prediction for the energy density of blackbody radiation) assuming that each mode has average energy kT.
\item[f)] Using Planks assumption of quantized energies, derive Plank's law $u(\lambda) = \frac{8\pi h c \lambda^{-5}}{e^{\frac{hc}{\lambda kT}} - 1}$.
Sketch this on top of the result from classical physics as a function of $\lambda$.
\textit{Hints:} $\sum_0^\infty x^n = \frac{1}{1-x}$ for $x<1$ and $\sum_0^\infty n e^{-nC} = - \frac{d}{dC} \sum_0^\infty e^{-nC}$
\item[g)] What experimental features of the black-body radiation can be explained by classical physics? Which cannot?
\item[h)] Show that Plank's law predicts the Stefan-Boltzmann law i.e. that the total energy density is proportional to $T^4$.
\item[i)] Show that Planck’s law predicts Wein’s law i.e. that $\lambda_{max} \sim \frac{1}{T}$.
\end{itemize}

\vspace*{0.5in}

%\textbf{4) Physics with Black body radiation}
%\begin{itemize}
%\item[-] 
%\item[-]A 40-W incandescent bulb radiates from a tungsten filament operating at 3300 K. 
%Assuming that the bulb radiates like a blackbody, what are the frequency and the wavelength at the maximum of the spectral distribution? 
%If the maximum frequency is a good approximation of the average frequency of the photons emitted by the bulb, about how many photons is the bulb radiating per second? 
%If you are looking at the bulb from 5 m away, how many photons enter your eye per second? (The diameter of your pupil is about 5.0 mm.)
%\end{itemize}

\textbf{4) Photoelectric Effect. } \hfill \textit{(15 points)}
\begin{itemize}
\item[a)]How is the result that the maximum photoelectric current is proportional to the intensity explained in the photon model of light?
\item[b)]What experimental features of the photoelectric effect can be explained by classical physics? Which cannot?
\item[c)]
This problem is one of \textit{estimating} the time lag (expected classically, but not observed) for the photoelectric effect. 
Assume that a point light source emits 1 W = 1 J/s of light energy.
Assuming uniform radiation in all directions, find the light intensity in $eV/(s \cdot m^2)$ at a distance of 1 m from the light source. 
Assuming some reasonable size for an atom, find the energy per unit time incident on the atom for this intensity. 
If the work function (see book for work function definition) is 2 eV, how long does it take for this much energy to be absorbed, assuming that all of the energy hitting the atom is absorbed?
\textbf{Estimate! Perfection not required. }
\end{itemize}


\clearpage

\textbf{5) Photons.  } \hfill \textit{(15 points)}
\begin{itemize}
\item[a)]
The wavelengths of visible light range from about 380 nm to about 750 nm. 
What is the range of photon energies (in eV) in visible light? 
A typical FM radio station’s broadcast frequency is about 100 MHz. 
What is the energy of an FM photon of the frequency?
\item[b)]
The NaCl molecule has a bond energy of 4.26 eV; that is, this energy must be supplied in order to dissociate the molecule into neutral Na and Cl atoms. 
What are the minimum frequency and maximum wavelength of the photon necessary to dissociate the molecule? 
In what part of the electromagnetic spectrum is this photon?
\item[c)]
A 40-W incandescent bulb radiates from a tungsten filament operating at 3300 K. 
Assuming that the bulb radiates like a blackbody, what are the frequency and the wavelength at the maximum of the spectral distribution? 
If the maximum frequency is a good approximation of the average frequency of the photons emitted by the bulb, about how many photons is the bulb radiating per second? 
If you are looking at the bulb from 5 m away, how many photons enter your eye per second? (The diameter of your pupil is about 5.0 mm.)
\end{itemize}



\end{document}
