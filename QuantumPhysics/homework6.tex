\documentclass[paper=letter,11pt]{scrartcl}

\KOMAoptions{headinclude=true, footinclude=false}
\KOMAoptions{DIV=14, BCOR=5mm}
\KOMAoptions{numbers=noendperiod}
\KOMAoptions{parskip=half}
\addtokomafont{disposition}{\rmfamily}
\addtokomafont{part}{\LARGE}
\addtokomafont{descriptionlabel}{\rmfamily}
%\setkomafont{pageheadfoot}{\normalsize\sffamily}
\setkomafont{pagehead}{\normalsize\rmfamily}
%\setkomafont{publishers}{\normalsize\rmfamily}
\setkomafont{caption}{\normalfont\small}
\setcapindent{0pt}
\deffootnote[1em]{1em}{1em}{\textsuperscript{\thefootnotemark}\ }


\usepackage{amsmath}
\usepackage[varg]{txfonts}
\usepackage[T1]{fontenc}
\usepackage{graphicx}
\usepackage{xcolor}
\usepackage[american]{babel}
% hyperref is needed in many places, so include it here
\usepackage{hyperref}

\usepackage{xspace}
\usepackage{multirow}
\usepackage{float}


\usepackage{braket}
\usepackage{bbm}
\usepackage{relsize}
\usepackage{tcolorbox}


\def\x{\ensuremath{x}}
\def\xp{\ensuremath{x'}}
\def\t{\ensuremath{t}}
\def\tp{\ensuremath{t'}}
\def\v{\ensuremath{v}}
%\def\nus{$\nu$s}

%\def\ketY{\ensuremath{\ket {\Psi}}}
%\def\iGeV{\ensuremath{\textrm{GeV}^{-1}}}
%%\def\mp{\ensuremath{m_{\textrm{proton}}}}
%\def\rp{\ensuremath{r_{\textrm{proton}}}}
%\def\me{\ensuremath{m_{\textrm{electron}}}}
%\def\aG{\ensuremath{\alpha_G}}
%\def\rAtom{\ensuremath{r_{\textrm{atom}}}}
%\def\rNucl{\ensuremath{r_{\textrm{nucleus}}}}
%\def\GN{\ensuremath{\textrm{G}_\textrm{N}}}
%\def\ketX{\ensuremath{\ket{\vec{x}}}}
%\def\ve{\ensuremath{\vec{\epsilon}}}
%
%
%\def\ABCDMatrix{\ensuremath{\begin{pmatrix} A &  B  \\ C  & D \end{pmatrix}}}
%\def\xyprime{\ensuremath{\begin{pmatrix} x' \\ y' \end{pmatrix}}}
%\def\xyprimeT{\ensuremath{\begin{pmatrix} x' &  y' \end{pmatrix}}}
%\def\xy{\ensuremath{\begin{pmatrix} x \\ y \end{pmatrix}}}
%\def\xyT{\ensuremath{\begin{pmatrix} x & y \end{pmatrix}}}
%
%\def\IMatrix{\ensuremath{\begin{pmatrix} 0 &  1  \\ -1  & 0 \end{pmatrix}}}
%\def\IBoostMatrix{\ensuremath{\begin{pmatrix} 0 &  1  \\ 1  & 0 \end{pmatrix}}}
%\def\JThree{\ensuremath{\begin{pmatrix}    0 & -i & 0  \\ i & 0  & 0 \\ 0 & 0 & 0 \end{pmatrix}}} 
%\def\JTwo{\ensuremath{\begin{bmatrix}    0 & 0 & -i  \\ 0 & 0  & 0 \\ i & 0 & 0 \end{bmatrix}}}
%\def\JOne{\ensuremath{\begin{bmatrix}    0 & 0 & 0  \\ 0 & 0  & -i \\ 0 & i & 0 \end{bmatrix}}}
%\def\etamn{\ensuremath{\eta_{\mu\nu}}}
%\def\Lmn{\ensuremath{\Lambda^\mu_\nu}}
%\def\dmn{\ensuremath{\delta^\mu_\nu}}
%\def\wmn{\ensuremath{\omega^\mu_\nu}}
%\def\be{\begin{equation*}}
%\def\ee{\end{equation*}}
%\def\bea{\begin{eqnarray*}}
%\def\eea{\end{eqnarray*}}
%\def\bi{\begin{itemize}}
%\def\ei{\end{itemize}}
%\def\fmn{\ensuremath{F_{\mu\nu}}}
%\def\fMN{\ensuremath{F^{\mu\nu}}}
%\def\bc{\begin{center}}
%\def\ec{\end{center}}
%\def\nus{$\nu$s}

\def\adagger{\ensuremath{a_{p\sigma}^\dagger}}
\def\lineacross{\noindent\rule{\textwidth}{1pt}}

\newcommand{\multiline}[1] {
\begin{tabular} {|l}
#1
\end{tabular}
}

\newcommand{\multilineNoLine}[1] {
\begin{tabular} {l}
#1
\end{tabular}
}



\newcommand{\lineTwo}[2] {
\begin{tabular} {|l}
#1 \\
#2
\end{tabular}
}

\newcommand{\rmt}[1] {
\textrm{#1}
}


%
% Units
%
\def\m{\ensuremath{\rmt{m}}}
\def\GeV{\ensuremath{\rmt{GeV}}}
\def\pt{\ensuremath{p_\rmt{T}}}


\def\parity{\ensuremath{\mathcal{P}}}

\usepackage{cancel}
\usepackage{ mathrsfs }
\def\bigL{\ensuremath{\mathscr{L}}}

\usepackage{ dsfont }

\def\nus{$\nu$s}
\def\nue{\ensuremath{\nu_e}}
\def\numu{\ensuremath{\nu_\mu}}
\def\nutau{\ensuremath{\nu_\tau}}
\def\nualpha{\ensuremath{\nu_\alpha}}
\def\nuone{\ensuremath{\nu_1}}
\def\nutwo{\ensuremath{\nu_2}}
\def\nuthree{\ensuremath{\nu_3}}


\usepackage{fancyhdr}
\fancyhf{}


\def\xyprime{\ensuremath{\begin{pmatrix} x' \\ y' \end{pmatrix}}}


\lhead{\Large 33-211} % \hfill Introduction to Particle Physics \hfill Spring 2022}
\chead{\Large Physics 3 : Modern Essentials} % \hfill Spring 2022}
\rhead{\Large Spring 2025} % \hfill Introduction to Particle Physics \hfill Spring 2022}
\begin{document}
\thispagestyle{fancy}





%\begin{tabular}{c}
%{\large 33-444 \hfill Intro To Particle \hfill Spring 2022\\}
%\hline 
%\end{tabular}

\begin{center}
{\huge \textbf{Homework Set \#6}}
\large

{\textbf{ Due Date:} Before class Friday March 21st  }
\end{center}

\textbf{1) Reading } \hfill \textit{(2 points)}\\
Read chapter 4.

\vspace*{0.25in}


{\large



\textbf{2) Compton Scattering } \hfill \textit{(15 points)}

Derive Compton's formula for relative shift of the wavelength of a photon scattered off an electron.
\begin{equation*}
\frac{\Delta \lambda}{\lambda} = \frac{\lambda_e}{\lambda}  (1 - \cos \theta) 
\end{equation*}
where, $\lambda$ is the wavelength of the incoming photon, $\Delta \lambda$ is the difference between the incoming and outgoing photon wavelengths,  $\theta$ is the angle between the direction incoming and outgoing photons and $\lambda_e = h/m_e c $ is the Compton wavelength of the electron.
What is the energy of a photon with wavelength equal to the Compton wavelength of the electron ?

Repeat for the case of a photon scattered of a proton. What is the numerical value for the ``Compton wavelength'' of the proton?

What is the energy of a photon with wavelength equal to the Compton wavelength of the proton?

\vspace*{0.5in}

\textbf{3) Rutherford.}  \hfill \textit{(15 points)}
\begin{itemize}
\item[(a)]
What was unexpected about Rutherford's alpha-ray scattering experiment? How did it qualitatively change our picture of atoms?
\item[(b)]
(Rutherford experiment in 1D) Suppose two particles with masses $m_\alpha$ and $m_N$ and initial velocities $v_\alpha$ and $v_N = 0$ along some line collide head on, emerging with velocities $v'_\alpha$ and $v'_N$ along the same line. Use conservation of momentum and energy to find $v'_\alpha$ as a function of $m_\alpha, m_N$, and $v_\alpha $.  If the alpha particle ``bounces back'' with significant velocity, eg: $v'_\alpha \sim - |v_\alpha|$, what does that say about the relative size of the masses $m_\alpha, m_N$ ?
\item[(c)]
What problems arise when applying classical physics to the ``solar system'' picture of atoms ? eg: what does the theory predict that is not observed?
\item[(d)]
Estimate an upper limit on the size of a gold nucleus from the distance of closest approach of a 1D head-on collision in which the incoming 8 MeV $\alpha$ particle returns with $\theta = 180^o$.
\end{itemize}

\clearpage

\textbf{4) Bohr.}  \hfill \textit{(15 points)}
\begin{itemize}
\item[(a)]
How are the sharp lines in atomic spectra explained in the Bohr model of the atom?
\item[(b)]
Show that the Bohr radius and the lowest energy for the hydrogen atom can be written as
\begin{equation*}
a_0 = \frac{hc}{\alpha mc^2} = \frac{\lambda_e}{2\pi \alpha} \hspace*{0.5in} E_1 = \frac{1}{2} \alpha^2 mc^2
\end{equation*}
where $\lambda_e = \frac{h}{mc}$ is the Compton wavelength of the electron and $\alpha = \frac{ke^2}{\hbar c}$ is the fine-structure constant. 
Use these expressions to check the numerical values of the constants $a_0$ and $E_1$.
\item[(c)] 
If the angular momentum of Earth in its motion around the Sun were quantized like a hydrogen electron according to Bohr's quantization hypothesis, what would Earth’s quantum number be? 
How much energy would be released in a transition to the next lowest level? 
Would that energy release (presumably as a gravity wave) be detectable? 
What would be the change in the radius of the earth's orbit ? 
(The radius of Earth’s orbit is $1.50 \times 10^{11}$ m.)
\item[(d)]
On average, a hydrogen atom will exist in an excited state for about $10^{-8}$ sec before making a transition to a lower energy state. 
About how many revolutions does an electron in the n=2 state make in $10^{-8}$ sec?
\end{itemize}




\end{document}
