\documentclass[paper=letter,11pt]{scrartcl}

\KOMAoptions{headinclude=true, footinclude=false}
\KOMAoptions{DIV=14, BCOR=5mm}
\KOMAoptions{numbers=noendperiod}
\KOMAoptions{parskip=half}
\addtokomafont{disposition}{\rmfamily}
\addtokomafont{part}{\LARGE}
\addtokomafont{descriptionlabel}{\rmfamily}
%\setkomafont{pageheadfoot}{\normalsize\sffamily}
\setkomafont{pagehead}{\normalsize\rmfamily}
%\setkomafont{publishers}{\normalsize\rmfamily}
\setkomafont{caption}{\normalfont\small}
\setcapindent{0pt}
\deffootnote[1em]{1em}{1em}{\textsuperscript{\thefootnotemark}\ }


\usepackage{amsmath}
\usepackage[varg]{txfonts}
\usepackage[T1]{fontenc}
\usepackage{graphicx}
\usepackage{xcolor}
\usepackage[american]{babel}
% hyperref is needed in many places, so include it here
\usepackage{hyperref}

\usepackage{xspace}
\usepackage{multirow}
\usepackage{float}


\usepackage{braket}
\usepackage{bbm}
\usepackage{relsize}
\usepackage{tcolorbox}


\def\x{\ensuremath{x}}
\def\xp{\ensuremath{x'}}
\def\t{\ensuremath{t}}
\def\tp{\ensuremath{t'}}
\def\v{\ensuremath{v}}
%\def\nus{$\nu$s}

%\def\ketY{\ensuremath{\ket {\Psi}}}
%\def\iGeV{\ensuremath{\textrm{GeV}^{-1}}}
%%\def\mp{\ensuremath{m_{\textrm{proton}}}}
%\def\rp{\ensuremath{r_{\textrm{proton}}}}
%\def\me{\ensuremath{m_{\textrm{electron}}}}
%\def\aG{\ensuremath{\alpha_G}}
%\def\rAtom{\ensuremath{r_{\textrm{atom}}}}
%\def\rNucl{\ensuremath{r_{\textrm{nucleus}}}}
%\def\GN{\ensuremath{\textrm{G}_\textrm{N}}}
%\def\ketX{\ensuremath{\ket{\vec{x}}}}
%\def\ve{\ensuremath{\vec{\epsilon}}}
%
%
%\def\ABCDMatrix{\ensuremath{\begin{pmatrix} A &  B  \\ C  & D \end{pmatrix}}}
%\def\xyprime{\ensuremath{\begin{pmatrix} x' \\ y' \end{pmatrix}}}
%\def\xyprimeT{\ensuremath{\begin{pmatrix} x' &  y' \end{pmatrix}}}
%\def\xy{\ensuremath{\begin{pmatrix} x \\ y \end{pmatrix}}}
%\def\xyT{\ensuremath{\begin{pmatrix} x & y \end{pmatrix}}}
%
%\def\IMatrix{\ensuremath{\begin{pmatrix} 0 &  1  \\ -1  & 0 \end{pmatrix}}}
%\def\IBoostMatrix{\ensuremath{\begin{pmatrix} 0 &  1  \\ 1  & 0 \end{pmatrix}}}
%\def\JThree{\ensuremath{\begin{pmatrix}    0 & -i & 0  \\ i & 0  & 0 \\ 0 & 0 & 0 \end{pmatrix}}} 
%\def\JTwo{\ensuremath{\begin{bmatrix}    0 & 0 & -i  \\ 0 & 0  & 0 \\ i & 0 & 0 \end{bmatrix}}}
%\def\JOne{\ensuremath{\begin{bmatrix}    0 & 0 & 0  \\ 0 & 0  & -i \\ 0 & i & 0 \end{bmatrix}}}
%\def\etamn{\ensuremath{\eta_{\mu\nu}}}
%\def\Lmn{\ensuremath{\Lambda^\mu_\nu}}
%\def\dmn{\ensuremath{\delta^\mu_\nu}}
%\def\wmn{\ensuremath{\omega^\mu_\nu}}
%\def\be{\begin{equation*}}
%\def\ee{\end{equation*}}
%\def\bea{\begin{eqnarray*}}
%\def\eea{\end{eqnarray*}}
%\def\bi{\begin{itemize}}
%\def\ei{\end{itemize}}
%\def\fmn{\ensuremath{F_{\mu\nu}}}
%\def\fMN{\ensuremath{F^{\mu\nu}}}
%\def\bc{\begin{center}}
%\def\ec{\end{center}}
%\def\nus{$\nu$s}

\def\adagger{\ensuremath{a_{p\sigma}^\dagger}}
\def\lineacross{\noindent\rule{\textwidth}{1pt}}

\newcommand{\multiline}[1] {
\begin{tabular} {|l}
#1
\end{tabular}
}

\newcommand{\multilineNoLine}[1] {
\begin{tabular} {l}
#1
\end{tabular}
}



\newcommand{\lineTwo}[2] {
\begin{tabular} {|l}
#1 \\
#2
\end{tabular}
}

\newcommand{\rmt}[1] {
\textrm{#1}
}


%
% Units
%
\def\m{\ensuremath{\rmt{m}}}
\def\GeV{\ensuremath{\rmt{GeV}}}
\def\pt{\ensuremath{p_\rmt{T}}}


\def\parity{\ensuremath{\mathcal{P}}}

\usepackage{cancel}
\usepackage{ mathrsfs }
\def\bigL{\ensuremath{\mathscr{L}}}

\usepackage{ dsfont }

\def\nus{$\nu$s}
\def\nue{\ensuremath{\nu_e}}
\def\numu{\ensuremath{\nu_\mu}}
\def\nutau{\ensuremath{\nu_\tau}}
\def\nualpha{\ensuremath{\nu_\alpha}}
\def\nuone{\ensuremath{\nu_1}}
\def\nutwo{\ensuremath{\nu_2}}
\def\nuthree{\ensuremath{\nu_3}}


\usepackage{fancyhdr}
\fancyhf{}


\def\xyprime{\ensuremath{\begin{pmatrix} x' \\ y' \end{pmatrix}}}


\lhead{\Large 33-211} % \hfill Introduction to Particle Physics \hfill Spring 2022}
\chead{\Large Physics 3 : Modern Essentials} % \hfill Spring 2022}
\rhead{\Large Spring 2023} % \hfill Introduction to Particle Physics \hfill Spring 2022}
\begin{document}
\thispagestyle{fancy}





%\begin{tabular}{c}
%{\large 33-444 \hfill Intro To Particle \hfill Spring 2022\\}
%\hline 
%\end{tabular}

\begin{center}
{\huge \textbf{Homework Set \#7}}
\large

{\textbf{ Due Date:} Before class Friday March 31st  }
\end{center}

\textbf{1) Reading } \hfill \textit{(2 points)}\\
Read chapter 5.


{\large
  
\textbf{2) Reduced Mass Correction. } \hfill \textit{(30 points)}
\begin{itemize}
\item[a)] Calculate the ratio of the wavelength of the transitions $n=2 \rightarrow n=1 $ in Hydrogen to the wavelength in Helium.
\item[b)] Derive a correction to the Bohr model which accounts for the motion of nucleus of the atom.  Let the position of the electron be $\vec{x}_e$ and nucleus be $\vec{x}_N$. Find the postion of the center of mass $\vec{x}_{CM}$ in terms of $\vec{x}_e$ and $\vec{x}_N$. Express $\vec{x}_e$ and $\vec{x}_N$ in terms of $\vec{x}_{CM}$ and $\vec{x}_{atom}$, where $\vec{x}_{atom}$ is the vector from the nucleus to the electron and use these to derive the momenta of the electron and nucleus. Now switch to the center of mass frame. What are $\vec{x}_e$ and $\vec{x}_N$ in this frame ? What is the total energy in this frame ? Calcluate the Bohr energies from this expression. 
\item[c)] Calculate the correction to a) from the result in b)
\item[d)] It is possible for a muon to be captured by a proton to form ``muononic hydrogen''. The muon is identical to the electron except it is 200 times heavier. Calculate the radius of the first Bohr orbit and the magnitude of the lowest energy state of muononic hydrogen.
\item[e)] It is possible for an electron and an anti-electron ``positron'' form a bound state ``positronium''. Calculate the radius of the first Bohr orbit and the magnitude of the lowest energy state  of positronium.
\end{itemize}

\textbf{3) Matter as Waves. }\hfill \textit{(15 points)}
\begin{itemize}
\item[(a)]
According to statistical mechanics, the average kinetic energy of a particle at temperature T is $\frac{3}{2}kT$, where k is the Boltzmann constant. 
What is the average de Broglie wavelength of nitrogen molecules at room temperature?

\item[(b)]
Find the de Broglie wavelength of a neutron of kinetic energy 0.02 eV (this is of the order of magnitude of kT at room temperature).

\item[(c)]
If the uncertainty in the position of a wave packet representing the state of a quantum-system particle is equal to its de Broglie wavelength, how does the uncertainty in momentum compare with the value of the momentum of the particle?

\item[(d)]
In order to locate a particle, e.g., an electron, to within $5 \times 10^{-12}$ m using electromagnetic waves (“light”), the wavelength must be at least this small. 
Calculate the momentum and energy of a photon with $\lambda = 5 \times 10^{-12}$ m. 
If the particle is an electron with $\Delta x = 5 \times 10^{-12}$ m, what is the corresponding uncertainty in its momentum?

\end{itemize}


\textbf{4) Wave packets. }\hfill \textit{(10 points)}
Show that if $y_n$ are solutions of the wave equation, then the fuction $y = \sum_n c_n y_n $ is also a solution for any values of the constants $c_n$


  

\end{document}
