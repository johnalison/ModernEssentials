\documentclass[paper=letter,11pt]{scrartcl}

\KOMAoptions{headinclude=true, footinclude=false}
\KOMAoptions{DIV=14, BCOR=5mm}
\KOMAoptions{numbers=noendperiod}
\KOMAoptions{parskip=half}
\addtokomafont{disposition}{\rmfamily}
\addtokomafont{part}{\LARGE}
\addtokomafont{descriptionlabel}{\rmfamily}
%\setkomafont{pageheadfoot}{\normalsize\sffamily}
\setkomafont{pagehead}{\normalsize\rmfamily}
%\setkomafont{publishers}{\normalsize\rmfamily}
\setkomafont{caption}{\normalfont\small}
\setcapindent{0pt}
\deffootnote[1em]{1em}{1em}{\textsuperscript{\thefootnotemark}\ }


\usepackage{amsmath}
\usepackage[varg]{txfonts}
\usepackage[T1]{fontenc}
\usepackage{graphicx}
\usepackage{xcolor}
\usepackage[american]{babel}
% hyperref is needed in many places, so include it here
\usepackage{hyperref}

\usepackage{xspace}
\usepackage{multirow}
\usepackage{float}


\usepackage{braket}
\usepackage{bbm}
\usepackage{relsize}
\usepackage{tcolorbox}


\def\x{\ensuremath{x}}
\def\xp{\ensuremath{x'}}
\def\t{\ensuremath{t}}
\def\tp{\ensuremath{t'}}
\def\v{\ensuremath{v}}
%\def\nus{$\nu$s}

%\def\ketY{\ensuremath{\ket {\Psi}}}
%\def\iGeV{\ensuremath{\textrm{GeV}^{-1}}}
%%\def\mp{\ensuremath{m_{\textrm{proton}}}}
%\def\rp{\ensuremath{r_{\textrm{proton}}}}
%\def\me{\ensuremath{m_{\textrm{electron}}}}
%\def\aG{\ensuremath{\alpha_G}}
%\def\rAtom{\ensuremath{r_{\textrm{atom}}}}
%\def\rNucl{\ensuremath{r_{\textrm{nucleus}}}}
%\def\GN{\ensuremath{\textrm{G}_\textrm{N}}}
%\def\ketX{\ensuremath{\ket{\vec{x}}}}
%\def\ve{\ensuremath{\vec{\epsilon}}}
%
%
%\def\ABCDMatrix{\ensuremath{\begin{pmatrix} A &  B  \\ C  & D \end{pmatrix}}}
%\def\xyprime{\ensuremath{\begin{pmatrix} x' \\ y' \end{pmatrix}}}
%\def\xyprimeT{\ensuremath{\begin{pmatrix} x' &  y' \end{pmatrix}}}
%\def\xy{\ensuremath{\begin{pmatrix} x \\ y \end{pmatrix}}}
%\def\xyT{\ensuremath{\begin{pmatrix} x & y \end{pmatrix}}}
%
%\def\IMatrix{\ensuremath{\begin{pmatrix} 0 &  1  \\ -1  & 0 \end{pmatrix}}}
%\def\IBoostMatrix{\ensuremath{\begin{pmatrix} 0 &  1  \\ 1  & 0 \end{pmatrix}}}
%\def\JThree{\ensuremath{\begin{pmatrix}    0 & -i & 0  \\ i & 0  & 0 \\ 0 & 0 & 0 \end{pmatrix}}} 
%\def\JTwo{\ensuremath{\begin{bmatrix}    0 & 0 & -i  \\ 0 & 0  & 0 \\ i & 0 & 0 \end{bmatrix}}}
%\def\JOne{\ensuremath{\begin{bmatrix}    0 & 0 & 0  \\ 0 & 0  & -i \\ 0 & i & 0 \end{bmatrix}}}
%\def\etamn{\ensuremath{\eta_{\mu\nu}}}
%\def\Lmn{\ensuremath{\Lambda^\mu_\nu}}
%\def\dmn{\ensuremath{\delta^\mu_\nu}}
%\def\wmn{\ensuremath{\omega^\mu_\nu}}
%\def\be{\begin{equation*}}
%\def\ee{\end{equation*}}
%\def\bea{\begin{eqnarray*}}
%\def\eea{\end{eqnarray*}}
%\def\bi{\begin{itemize}}
%\def\ei{\end{itemize}}
%\def\fmn{\ensuremath{F_{\mu\nu}}}
%\def\fMN{\ensuremath{F^{\mu\nu}}}
%\def\bc{\begin{center}}
%\def\ec{\end{center}}
%\def\nus{$\nu$s}

\def\adagger{\ensuremath{a_{p\sigma}^\dagger}}
\def\lineacross{\noindent\rule{\textwidth}{1pt}}

\newcommand{\multiline}[1] {
\begin{tabular} {|l}
#1
\end{tabular}
}

\newcommand{\multilineNoLine}[1] {
\begin{tabular} {l}
#1
\end{tabular}
}



\newcommand{\lineTwo}[2] {
\begin{tabular} {|l}
#1 \\
#2
\end{tabular}
}

\newcommand{\rmt}[1] {
\textrm{#1}
}


%
% Units
%
\def\m{\ensuremath{\rmt{m}}}
\def\GeV{\ensuremath{\rmt{GeV}}}
\def\pt{\ensuremath{p_\rmt{T}}}


\def\parity{\ensuremath{\mathcal{P}}}

\usepackage{cancel}
\usepackage{ mathrsfs }
\def\bigL{\ensuremath{\mathscr{L}}}

\usepackage{ dsfont }

\def\nus{$\nu$s}
\def\nue{\ensuremath{\nu_e}}
\def\numu{\ensuremath{\nu_\mu}}
\def\nutau{\ensuremath{\nu_\tau}}
\def\nualpha{\ensuremath{\nu_\alpha}}
\def\nuone{\ensuremath{\nu_1}}
\def\nutwo{\ensuremath{\nu_2}}
\def\nuthree{\ensuremath{\nu_3}}


\usepackage{fancyhdr}
\fancyhf{}


\def\xyprime{\ensuremath{\begin{pmatrix} x' \\ y' \end{pmatrix}}}


\lhead{\Large 33-211} % \hfill Introduction to Particle Physics \hfill Spring 2022}
\chead{\Large Physics 3 : Modern Essentials} % \hfill Spring 2022}
\rhead{\Large Spring 2025} % \hfill Introduction to Particle Physics \hfill Spring 2022}
\begin{document}
\thispagestyle{fancy}





%\begin{tabular}{c}
%{\large 33-444 \hfill Intro To Particle \hfill Spring 2022\\}
%\hline 
%\end{tabular}

\begin{center}
{\huge \textbf{Homework Set \#9}}
\large

{\textbf{ Due Date:} Before class Friday April 25th  }
\end{center}

{\large


\textbf{1) Dynamics. }\hfill \textit{(10 points)}
\begin{itemize}
\item[(a)]{ 
Suppose a particle starts out in a linear combination of two stationary states: 

\begin{equation}
\Psi(x,0) = c_1 \psi_1(x) + c_2 \psi_2(x)
\end{equation}

with energies $E_1$ and $E_2$ where, for simplicity,  $c_1$ and $c_2$ are taken to be real.
What is the wave function $\Psi(x,t)$ at later times ?
Find the probability density, and describe its motion.
}
\end{itemize}

\vspace*{0.4in}

\textbf{2) Infinite Square Well. }\hfill \textit{(30 points)}
\begin{itemize}
\item[(a)]{
In the early days of nuclear physics, before the neutron was discovered, it was thought that the nucleus contained only electrons and protons. 
If we consider the nucleus to be a one-dimensional infinite well with L = 10 fm, compute the ground-state energy for (1) an electron and (2) a proton in the nucleus. 
Compute the energy difference between the ground state and the first excited state for each particle.
Differences between energy levels in nuclei are observed to be typically of the order of 1 MeV.
Comment on any apparent problems with this model.
}
\item[(b)]{
The wavelength of light emitted by a ruby laser is 694.3 nm.
Assuming that the emission of a photon of this wavelength accompanies the transition of an electron from the n = 2 level to the n = 1 level of an infinite square well, compute L for the well.
}
\item[(c)]{
Show directly from the time-independent Schrodinger equation that $\left<p^2\right> = \left<2m[E - V(x)]\right>$. 
Use this result to compute $\left<p^2\right>$ for the ground state of the infinite square well.
}
\item[(d)]{
Find $\sigma_x$ and $\sigma_p$ and the product $\sigma_x\cdot\sigma_p$ for the ground-state wave function of an infinite square well.
Comment on the result in relation to the uncertainty principle. 
(Feel free to look up any integrals you may need.)
}

\end{itemize}

\clearpage

\textbf{3) Many Particles. }\hfill \textit{ 15 Points}

Consider a 1D infinite well of length L.
What is the ground state wave function \textbf{and} energy for:
\begin{itemize}
\item[-]{Two distinguishable particles?}
\vspace*{0.2in}
\item[-]{Two identical Bosons ?}
\vspace*{0.2in}
\item[-]{Two identical Fermions ?}
\vspace*{0.2in}
\end{itemize}


\textbf{4) Molecules. }\hfill \textit{ 20 Points}
\begin{itemize}
\item[-] Use the infinite well as a model to analyze: % 5
\begin{equation}
H + H \rightarrow H_2
\end{equation}
Treat $H$ as infinite well of lentgh L and $H_2$ and an infinite well of length 3/2 L.\\
Do you expect $H_2$ to be stable ? Why, why not?
\item[-] Use the infinite well as a model to analyze:  % 5
\begin{equation}
He + He \rightarrow He_2 
\end{equation}
Again, treat $He$ as infinite well of lentgh L and $He_2$ and an infinite well of length 3/2 L. \\
Do you expect $He_2$ to be stable ? Why, why not?
\end{itemize}


\clearpage

\end{document}
