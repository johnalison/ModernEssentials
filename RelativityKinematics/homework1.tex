\documentclass[paper=letter,11pt]{scrartcl}

\KOMAoptions{headinclude=true, footinclude=false}
\KOMAoptions{DIV=14, BCOR=5mm}
\KOMAoptions{numbers=noendperiod}
\KOMAoptions{parskip=half}
\addtokomafont{disposition}{\rmfamily}
\addtokomafont{part}{\LARGE}
\addtokomafont{descriptionlabel}{\rmfamily}
%\setkomafont{pageheadfoot}{\normalsize\sffamily}
\setkomafont{pagehead}{\normalsize\rmfamily}
%\setkomafont{publishers}{\normalsize\rmfamily}
\setkomafont{caption}{\normalfont\small}
\setcapindent{0pt}
\deffootnote[1em]{1em}{1em}{\textsuperscript{\thefootnotemark}\ }


\usepackage{amsmath}
\usepackage[varg]{txfonts}
\usepackage[T1]{fontenc}
\usepackage{graphicx}
\usepackage{xcolor}
\usepackage[american]{babel}
% hyperref is needed in many places, so include it here
\usepackage{hyperref}

\usepackage{xspace}
\usepackage{multirow}
\usepackage{float}


\usepackage{braket}
\usepackage{bbm}
\usepackage{relsize}
\usepackage{tcolorbox}


\def\x{\ensuremath{x}}
\def\xp{\ensuremath{x'}}
\def\t{\ensuremath{t}}
\def\tp{\ensuremath{t'}}
\def\v{\ensuremath{v}}
%\def\nus{$\nu$s}

%\def\ketY{\ensuremath{\ket {\Psi}}}
%\def\iGeV{\ensuremath{\textrm{GeV}^{-1}}}
%%\def\mp{\ensuremath{m_{\textrm{proton}}}}
%\def\rp{\ensuremath{r_{\textrm{proton}}}}
%\def\me{\ensuremath{m_{\textrm{electron}}}}
%\def\aG{\ensuremath{\alpha_G}}
%\def\rAtom{\ensuremath{r_{\textrm{atom}}}}
%\def\rNucl{\ensuremath{r_{\textrm{nucleus}}}}
%\def\GN{\ensuremath{\textrm{G}_\textrm{N}}}
%\def\ketX{\ensuremath{\ket{\vec{x}}}}
%\def\ve{\ensuremath{\vec{\epsilon}}}
%
%
%\def\ABCDMatrix{\ensuremath{\begin{pmatrix} A &  B  \\ C  & D \end{pmatrix}}}
%\def\xyprime{\ensuremath{\begin{pmatrix} x' \\ y' \end{pmatrix}}}
%\def\xyprimeT{\ensuremath{\begin{pmatrix} x' &  y' \end{pmatrix}}}
%\def\xy{\ensuremath{\begin{pmatrix} x \\ y \end{pmatrix}}}
%\def\xyT{\ensuremath{\begin{pmatrix} x & y \end{pmatrix}}}
%
%\def\IMatrix{\ensuremath{\begin{pmatrix} 0 &  1  \\ -1  & 0 \end{pmatrix}}}
%\def\IBoostMatrix{\ensuremath{\begin{pmatrix} 0 &  1  \\ 1  & 0 \end{pmatrix}}}
%\def\JThree{\ensuremath{\begin{pmatrix}    0 & -i & 0  \\ i & 0  & 0 \\ 0 & 0 & 0 \end{pmatrix}}} 
%\def\JTwo{\ensuremath{\begin{bmatrix}    0 & 0 & -i  \\ 0 & 0  & 0 \\ i & 0 & 0 \end{bmatrix}}}
%\def\JOne{\ensuremath{\begin{bmatrix}    0 & 0 & 0  \\ 0 & 0  & -i \\ 0 & i & 0 \end{bmatrix}}}
%\def\etamn{\ensuremath{\eta_{\mu\nu}}}
%\def\Lmn{\ensuremath{\Lambda^\mu_\nu}}
%\def\dmn{\ensuremath{\delta^\mu_\nu}}
%\def\wmn{\ensuremath{\omega^\mu_\nu}}
%\def\be{\begin{equation*}}
%\def\ee{\end{equation*}}
%\def\bea{\begin{eqnarray*}}
%\def\eea{\end{eqnarray*}}
%\def\bi{\begin{itemize}}
%\def\ei{\end{itemize}}
%\def\fmn{\ensuremath{F_{\mu\nu}}}
%\def\fMN{\ensuremath{F^{\mu\nu}}}
%\def\bc{\begin{center}}
%\def\ec{\end{center}}
%\def\nus{$\nu$s}

\def\adagger{\ensuremath{a_{p\sigma}^\dagger}}
\def\lineacross{\noindent\rule{\textwidth}{1pt}}

\newcommand{\multiline}[1] {
\begin{tabular} {|l}
#1
\end{tabular}
}

\newcommand{\multilineNoLine}[1] {
\begin{tabular} {l}
#1
\end{tabular}
}



\newcommand{\lineTwo}[2] {
\begin{tabular} {|l}
#1 \\
#2
\end{tabular}
}

\newcommand{\rmt}[1] {
\textrm{#1}
}


%
% Units
%
\def\m{\ensuremath{\rmt{m}}}
\def\GeV{\ensuremath{\rmt{GeV}}}
\def\pt{\ensuremath{p_\rmt{T}}}


\def\parity{\ensuremath{\mathcal{P}}}

\usepackage{cancel}
\usepackage{ mathrsfs }
\def\bigL{\ensuremath{\mathscr{L}}}

\usepackage{ dsfont }

\def\nus{$\nu$s}
\def\nue{\ensuremath{\nu_e}}
\def\numu{\ensuremath{\nu_\mu}}
\def\nutau{\ensuremath{\nu_\tau}}
\def\nualpha{\ensuremath{\nu_\alpha}}
\def\nuone{\ensuremath{\nu_1}}
\def\nutwo{\ensuremath{\nu_2}}
\def\nuthree{\ensuremath{\nu_3}}


\usepackage{fancyhdr}
\fancyhf{}




\lhead{\Large 33-211} % \hfill Introduction to Particle Physics \hfill Spring 2022}
\chead{\Large Physics 3 : Modern Essentials} % \hfill Spring 2022}
\rhead{\Large Spring 2025} % \hfill Introduction to Particle Physics \hfill Spring 2022}
\begin{document}
\thispagestyle{fancy}





%\begin{tabular}{c}
%{\large 33-444 \hfill Intro To Particle \hfill Spring 2022\\}
%\hline 
%\end{tabular}

\begin{center}
{\huge \textbf{Homework Set \#1}}
\large

{\textbf{ Due Date:} Before class Friday January 24th  }
\end{center}

{\large
\textbf{1) You} \hfill \textit{(2 points)}\\
\begin{itemize}
\item[(a)]What is your major(s)/minor(s) ? 
\item[(b)]Have you studied Matrices ? Relativity ? or QM before?
\item[(c)]What do you most want to get out of this course ? 
\end{itemize}

\vspace*{0.25in}

\textbf{2) Reading } \hfill \textit{(2 points)}\\
\begin{itemize}
\item[-] Read sections 1-1,1-2,1-3,1-4
\end{itemize}

\vspace*{0.25in}


\textbf{3) Matrix Multiplication} \hfill \textit{(2 points)}\\

Let, $A =  \begin{bmatrix}    a_{11} & a_{12}  \\ a_{21} & a_{22} \end{bmatrix}$
     $B =  \begin{bmatrix}    b_{11} & b_{12}  \\ b_{21} & b_{22} \end{bmatrix}$
     $C =  \begin{bmatrix}    c_{1}  & \\ c_{2} & \end{bmatrix}$

\begin{itemize}
\item[(a)] What is A$\times$ C ? 
\item[(b)] What is A$\times$ B ?
\item[(b)] What is n$\times$ A ? where n is a number. 
\item[(c)] Watch this:  \href{https://www.youtube.com/watch?v=kYB8IZa5AuE&list=PLZHQObOWTQDPD3MizzM2xVFitgF8hE_ab&index=3}{The adult way to think about matrices.}
\end{itemize}

\vspace*{0.25in}


%\textbf{4) Neutron Stars } \hfill \textit{(5 points)}
%\begin{itemize}
%\item[(a)]Estimate the radius, mass, and speed of sound for neutron stars in terms of $\alpha$, \aG, \mp, and \me.
%Assume that a neutron star is a solid made of neutrons and $\mp \sim m_{\textrm{neutron}}$. \\
%(Hint: the speed of sound is given by the square-root of the pressure over the mass density)
%\item[(b)]What are you estimated values in mks units ?
%\item[(c)]Compare your estimates to actual values for Neutron Stars quoted online.
%\item[(d)]Look up $m_{\textrm{neutron}}$. How does this compare with the assumption of $\mp \sim m_{\textrm{neutron}}$?
%\end{itemize}

\textbf{3) Galilean Transformations } \hfill \textit{(10 points)}\\

Consider an isolated system of N particles. The total momentum is given by $P_{\textrm{total}} = \sum_i P_i$, where the momentum of the ith particle is $m_i v_i$ and the total energy by $E_{\textrm{total}} = \sum_i E_i$, where the energy of the ith particle is $1/2 m_i v_i^2$
\begin{itemize}
\item[(a)]Show that if the total momentum is conserved in one reference frame (S) it is also conserved in a reference frame (S`) moving with velocity $v$ with respect to the first frame.
\item[(b)]Show that if energy is conserved in reference frame S it is also conserved in reference frame (S`)
\end{itemize}
}

\vspace*{0.25in}

\textbf{4) General Linear Coordinate Transformations } \hfill \textit{(35 points)}\\

Fill in the details for argument for the general coordinate transformation sketched in class.
Write down the arbitrary linear transformation we assumed in class.
The unknowns are arbitrary functions of \v, but not (by assumption \x,\xp,\t, or \tp).

\begin{itemize}
\item[(a)] Impose the constraint \x=\v\t\ when \xp=0. What is the most general form now?
\item[(b)] Impose the constraint \xp=-\v\tp\ when \x=0. What is the most general form now?
\item[(c)] Require that the combination of two transforms ($v_1$) and ($v_2$) yields another transform. (Hint: What relation do the diagonal elements have to have to form a valid transformation consistent with b)? ) (Hint \#2 This should yield a free parameter given by the separation of variables constant)
\item[(d)] You should now have one free parameter (the separation constant) and one unknown arbitrary function of v. Solve for the arbitrary function by requiring the two transforms ($v$) and ($-v$) to give the identity (ie: x'=x and t'=t)
\item[(e)]You should now have the most general linear coordinate transformation.  Write the separation constant in terms of a velocity $v_*$ as we did in class. Show that if something is moving with $v_*$ in one frame it move with the speed in the other frame. (ie: $\xp=v_*\tp \implies \x=v_*\t$ )
\item[(f)]Show that $v_*^2t^2 - x^2$ is invariant under coordinate transformations.
\item[(h)]How does the most general coordinate transformation relate to the Galilean transformations in classical physics? eg: What choice does Newton make for $v_*$ ? A pithy way to summarize the difference between Newtonian physics and Relativistic physics is simply a different choice for the free parameter in the general coordinate transformation. 
\end{itemize}

\vspace*{0.25in}

\textbf{5) Proton Collisions in Newtonian Mechanics  } \hfill \textit{(10 points)}\\

A proton moving with speed $v_i$ strikes a proton at rest in the lab frame.
After the collision the protons are observed to scatter symmetrically about the x-axis.
(ie: one proton makes an angle +$\theta$ with respect to the x-axis, the other makes an angle -$\theta$. )
What does Newtonian physics predict the angle between the protons should be in the lab frame ?

\vspace*{0.25in}

\textbf{6) How long does it take light to travel a foot ?} \hfill \textit{(2 points)}\\


\clearpage
\textbf{7) What is an inertial frame ? } \hfill \textit{(6 points)}\\
\begin{itemize}
\item[-]{ Are you in an internal frame when riding on a smooth high speed train ? Why or Why not?}
\item[-]{ Are you in an internal frame when riding on a merry-go-round ? Why or Why not?}
\end{itemize}

\vspace*{0.25in}


\textbf{8) Two runners } \hfill \textit{(5 points)}\\

Two runners (A and B) are racing along a straight track of length L. B travels at a constant speed \v, A goes half of the way with speed 2\v\ and half of the way with speed v/2. Who wins?  

\vspace*{0.25in}

\textbf{9) State the postulate(s) of Special Relativity.} \hfill \textit{(2 points)}\\

\vspace*{0.25in}

\textbf{10) Different kinds of clocks. } \hfill \textit{(5 points)}\\

There are many different kinds of clocks: sand clocks, electric clocks, mechanical clocks, light clocks and biological clocks.
We have shown that  light clocks in motion slow down,  does this  necessarily imply that  all  clocks will be effected equally?


\end{document}
