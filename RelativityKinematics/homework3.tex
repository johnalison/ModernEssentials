\documentclass[paper=letter,11pt]{scrartcl}

\KOMAoptions{headinclude=true, footinclude=false}
\KOMAoptions{DIV=14, BCOR=5mm}
\KOMAoptions{numbers=noendperiod}
\KOMAoptions{parskip=half}
\addtokomafont{disposition}{\rmfamily}
\addtokomafont{part}{\LARGE}
\addtokomafont{descriptionlabel}{\rmfamily}
%\setkomafont{pageheadfoot}{\normalsize\sffamily}
\setkomafont{pagehead}{\normalsize\rmfamily}
%\setkomafont{publishers}{\normalsize\rmfamily}
\setkomafont{caption}{\normalfont\small}
\setcapindent{0pt}
\deffootnote[1em]{1em}{1em}{\textsuperscript{\thefootnotemark}\ }


\usepackage{amsmath}
\usepackage[varg]{txfonts}
\usepackage[T1]{fontenc}
\usepackage{graphicx}
\usepackage{xcolor}
\usepackage[american]{babel}
% hyperref is needed in many places, so include it here
\usepackage{hyperref}

\usepackage{xspace}
\usepackage{multirow}
\usepackage{float}


\usepackage{braket}
\usepackage{bbm}
\usepackage{relsize}
\usepackage{tcolorbox}


\def\x{\ensuremath{x}}
\def\xp{\ensuremath{x'}}
\def\t{\ensuremath{t}}
\def\tp{\ensuremath{t'}}
\def\v{\ensuremath{v}}
%\def\nus{$\nu$s}

%\def\ketY{\ensuremath{\ket {\Psi}}}
%\def\iGeV{\ensuremath{\textrm{GeV}^{-1}}}
%%\def\mp{\ensuremath{m_{\textrm{proton}}}}
%\def\rp{\ensuremath{r_{\textrm{proton}}}}
%\def\me{\ensuremath{m_{\textrm{electron}}}}
%\def\aG{\ensuremath{\alpha_G}}
%\def\rAtom{\ensuremath{r_{\textrm{atom}}}}
%\def\rNucl{\ensuremath{r_{\textrm{nucleus}}}}
%\def\GN{\ensuremath{\textrm{G}_\textrm{N}}}
%\def\ketX{\ensuremath{\ket{\vec{x}}}}
%\def\ve{\ensuremath{\vec{\epsilon}}}
%
%
%\def\ABCDMatrix{\ensuremath{\begin{pmatrix} A &  B  \\ C  & D \end{pmatrix}}}
%\def\xyprime{\ensuremath{\begin{pmatrix} x' \\ y' \end{pmatrix}}}
%\def\xyprimeT{\ensuremath{\begin{pmatrix} x' &  y' \end{pmatrix}}}
%\def\xy{\ensuremath{\begin{pmatrix} x \\ y \end{pmatrix}}}
%\def\xyT{\ensuremath{\begin{pmatrix} x & y \end{pmatrix}}}
%
%\def\IMatrix{\ensuremath{\begin{pmatrix} 0 &  1  \\ -1  & 0 \end{pmatrix}}}
%\def\IBoostMatrix{\ensuremath{\begin{pmatrix} 0 &  1  \\ 1  & 0 \end{pmatrix}}}
%\def\JThree{\ensuremath{\begin{pmatrix}    0 & -i & 0  \\ i & 0  & 0 \\ 0 & 0 & 0 \end{pmatrix}}} 
%\def\JTwo{\ensuremath{\begin{bmatrix}    0 & 0 & -i  \\ 0 & 0  & 0 \\ i & 0 & 0 \end{bmatrix}}}
%\def\JOne{\ensuremath{\begin{bmatrix}    0 & 0 & 0  \\ 0 & 0  & -i \\ 0 & i & 0 \end{bmatrix}}}
%\def\etamn{\ensuremath{\eta_{\mu\nu}}}
%\def\Lmn{\ensuremath{\Lambda^\mu_\nu}}
%\def\dmn{\ensuremath{\delta^\mu_\nu}}
%\def\wmn{\ensuremath{\omega^\mu_\nu}}
%\def\be{\begin{equation*}}
%\def\ee{\end{equation*}}
%\def\bea{\begin{eqnarray*}}
%\def\eea{\end{eqnarray*}}
%\def\bi{\begin{itemize}}
%\def\ei{\end{itemize}}
%\def\fmn{\ensuremath{F_{\mu\nu}}}
%\def\fMN{\ensuremath{F^{\mu\nu}}}
%\def\bc{\begin{center}}
%\def\ec{\end{center}}
%\def\nus{$\nu$s}

\def\adagger{\ensuremath{a_{p\sigma}^\dagger}}
\def\lineacross{\noindent\rule{\textwidth}{1pt}}

\newcommand{\multiline}[1] {
\begin{tabular} {|l}
#1
\end{tabular}
}

\newcommand{\multilineNoLine}[1] {
\begin{tabular} {l}
#1
\end{tabular}
}



\newcommand{\lineTwo}[2] {
\begin{tabular} {|l}
#1 \\
#2
\end{tabular}
}

\newcommand{\rmt}[1] {
\textrm{#1}
}


%
% Units
%
\def\m{\ensuremath{\rmt{m}}}
\def\GeV{\ensuremath{\rmt{GeV}}}
\def\pt{\ensuremath{p_\rmt{T}}}


\def\parity{\ensuremath{\mathcal{P}}}

\usepackage{cancel}
\usepackage{ mathrsfs }
\def\bigL{\ensuremath{\mathscr{L}}}

\usepackage{ dsfont }

\def\nus{$\nu$s}
\def\nue{\ensuremath{\nu_e}}
\def\numu{\ensuremath{\nu_\mu}}
\def\nutau{\ensuremath{\nu_\tau}}
\def\nualpha{\ensuremath{\nu_\alpha}}
\def\nuone{\ensuremath{\nu_1}}
\def\nutwo{\ensuremath{\nu_2}}
\def\nuthree{\ensuremath{\nu_3}}


\usepackage{fancyhdr}
\fancyhf{}


\def\xyprime{\ensuremath{\begin{pmatrix} x' \\ y' \end{pmatrix}}}


\lhead{\Large 33-211} % \hfill Introduction to Particle Physics \hfill Spring 2022}
\chead{\Large Physics 3 : Modern Essentials} % \hfill Spring 2022}
\rhead{\Large Spring 2025} % \hfill Introduction to Particle Physics \hfill Spring 2022}
\begin{document}
\thispagestyle{fancy}





%\begin{tabular}{c}
%{\large 33-444 \hfill Intro To Particle \hfill Spring 2022\\}
%\hline 
%\end{tabular}

\begin{center}
{\huge \textbf{Homework Set \#3}}
\large

{\textbf{ Due Date:} Before class Friday February 14th  }
\end{center}

\textbf{1) Reading } \hfill \textit{(2 points)}\\
Read chapter 2.

\vspace*{0.25in}


{\large
\textbf{2) Acceleration } \hfill \textit{(8 points)}\\
Derive the transformation of the x and y components of acceleration.

\vspace*{0.25in}


\textbf{3) Doppler Effect } \hfill \textit{(10 points)}\\
Work out the relativistic Doppler effect for the case when the source is approaching the receiver.

\vspace*{0.25in}

\textbf{4) Velocity Parameter  } \hfill \textit{(5 points)}\\
Show that the Lorentz transforms as expressed in terms of the velocity parameter ($\eta$) leaves the $t^2-x^2$ invariant.

\vspace*{0.25in}


\textbf{5) Relative Velocity Parameter  } \hfill \textit{(5 points)}\\
A particle moves with speed 0.9c along the x'' axis of frame S'' which moves with speed 0.9c in the positive x' direction relative to frame S'.
Frame S' moves with speed 0.9c in the positive x direction relative to frame S. Find the speed of the particle relative to frame S.
Solve using the velocity parameter.

\vspace*{0.25in}

\textbf{6) Calibrating Space-time diagrams  } \hfill \textit{(15 points)}\\
When calibrating the x-coordinate in spacetime diagrams with the invariant hyperbola, show that the time axis for any frame is parallel to the tangent of the hyperbola drawn at the point where the hyperbola intersects the corresponding x axis.

\vspace*{0.25in}

\clearpage

\textbf{7) Space-time diagrams  } \hfill \textit{(15 points)}\\
Two inertial coordinate systems S and S' move with speed c/2 with respect to each other.
Draw a  space-time diagram relating these two systems (Let the axes of x and t for S be at right angles in your drawing)

\begin{itemize}
\item[a)] Draw calibration hyperbolas that allow you to define unit distances along the axes of x, x' t and t'.
\item[b)] Plot the following events on the diagram 1) (x,t) = (1,1); 2) (x',t')=(1,1); 3) (x',t')=(2,0); 4) (x',t')=(0,2);
\item[c)] Determine the coordinate in S' (or S) for the corresponding to the events in b)
\end{itemize}

\vspace*{0.25in}

\textbf{8) Relativistic Proton Collisions  } \hfill \textit{(30 points)}\\
Proton A collides elastically with proton B, which is at rest.
The outcome of each individual collision cannot be predicted.
However, occasionally there is a ``symmetric collision'' in which the two protons come off with identical speeds along paths that make identical angles $\alpha_A = \alpha_B = \alpha/2$ with the forward direction.
What is this angle of deflection in a symmetric collision?
For Newtonian mechanics the total angle of separation is always 90$^o$. (Review your solution to problem 5 in homework \# 1.)
You will see that this angle will be less than 90$^o$ for a relativistic impact.
This is one of the most decisive predictions that confirms relativity.
The difference between the separation angle from 90$^o$ provides a useful measure of the departure from Newtonian mechanics.
%Show that the incident proton can have a velocity as high as $\beta = 2/7$ without making the angle between the outgoing protons in teh lab frame depart from 90$^o$ by as much as 0.01 radian.  
How high must the velocity of proton A be before the separation angle deviates from 90$^o$ by as much as 1/100 of a radian?
Compare your answer to problem 6 of homework \#2.

(Hint 1: It simplifies the analysis to use a frame of reference in which the collision is symmetric.)
(Hint 2: You need to calculate the angle between to \textbf{velocities} in the lab frame.)







\end{document}
