\documentclass[margin,line]{res}
\usepackage{amsmath}

%\topmargin -.5in
\oddsidemargin -.5in
\evensidemargin -.5in
\textwidth=6.0in
\itemsep=0in
\parsep=0in

\newenvironment{list1}{
  \begin{list}{\ding{113}}{%
      \setlength{\itemsep}{0in}
      \setlength{\parsep}{0in} \setlength{\parskip}{0in}
      \setlength{\topsep}{0in} \setlength{\partopsep}{0in} 
      \setlength{\leftmargin}{0.17in}}}{\end{list}}
\newenvironment{list2}{
  \begin{list}{$\bullet$}{%
      \setlength{\itemsep}{0in}
      \setlength{\parsep}{0in} \setlength{\parskip}{0in}
      \setlength{\topsep}{0in} \setlength{\partopsep}{0in} 
      \setlength{\leftmargin}{0.2in}}}{\end{list}}

\usepackage{amssymb}
\renewcommand{\labelitemi}{$\bullet$}


\newcommand{\MYhref}[3][blue]{\href{#2}{\color{#1}{#3}}}%
\usepackage{xcolor}
\usepackage[colorlinks = true,
            linkcolor = blue,
            urlcolor  = blue,
            citecolor = blue,
            anchorcolor = blue]{hyperref}


\newcommand{\litem}{\item[\Large\textbf{-}]}

\usepackage{etoolbox}
\newtoggle{isCV}
\toggletrue{isCV}


\begin{document}

\name{\huge Physics-III: Modern Essentials (33-211)  \vspace*{.1in}}
%\name{\huge Revolution in Physics: Space-time and Quantum Physics (33-211) \vspace*{.1in}}

%\name{\huge Revolution in Physics: Space-time and Quantum Physics (33-211) \vspace*{.1in}}


\begin{resume}
{ \textit{\large``Not only God knows, I know, and by the end of the semester, you will know.''}\\ \hspace*{5in} -Sidney Coleman}

\section{Course Description:}

This course is about the revolutions in physics (Relativity and Quantum Mechanics) that took place in the early 20th century.
Relativity and Quantum Mechanics changed our understanding of reality and what we take physics to be.
They were important both philosophically and practically in shaping the modern world.
Philosophically, they up-ended our common-sense notion of the way the world works.
Practically, they underlie our understanding of everything from the basic stability of matter to the operation of modern electronics.
These theories are the foundation for modern physics;
the Standard Model of particle physics -- our current best understanding of nature on the smallest scales -- is by and large a direct consequence of the combination of the principles Relativity and Quantum Mechanics.
These revolutions continue to take center stage in the drama that underlies current research in physics and cosmology.

The first part of the course will focus on the motivation for -- and implications of -- Einstein's theory of relativity.
Relativity is the correct way to think about space and time.
The necessary mathematics is simple, but the concepts challenge our everyday intuition.

The second part of the course introduces Quantum Mechanics, the basic theory used to describe the microscopic world.
We will briefly discuss its historical development and how -- by completely changing our view of reality -- we were finally able to account for basic features of our everyday lives.
%We will introduce the duality between wave-like and particle-like phenomena and develop the wave equation of quantum mechanics.
Here again, the concepts are challenging, but the mathematical formalism is straightforward.

The class will improve your skill in thinking through and solving physics problems and apparent philosophical paradoxes.
One of the main goals of the course is to develop reliable intuition for relativistic and quantum physics.



\section{Professor:}
John Alison\\
Office: 7420 Wean\\
Email: johnalison@cmu.edu

\section{Lectures:}
MWF 9:00-9:50 am in Doherty Hall 1211 \\
H  10:00-10:50 am in Wean Hall 7316


\section{Office Hours:}
T: 3:30 - 4:30\\
H: 2:30 - 3:30

\section{References:}
This course use the textbook “Modern Physics, 5th edition”, by Paul A. Tipler and Ralph A. Llewellyn, as well as some extra lecture material and supplementary notes.
This book has been chosen because it covers the required material at an appropriate level and is available electronically online.
Message me if you cannot find a copy with a simple web search.

\section{Course Website:}  
\href{https://canvas.cmu.edu/courses/32799}{https://canvas.cmu.edu/courses/32799}

\section{Grader:}
Patrick Shaw  \\
Email: pshaw2@andrew.cmu.edu

\clearpage

\section{Grading:} 
There will be three one-hour exams during the semester and a three-hour comprehensive final exam.
The hour exams will be given during regular class times in the usual lecture room.
They will be closed-book exams with no electronic devices permitted.
If you cannot take an exam for any reason, tell the instructor beforehand, after-the-fact make-up exams cannot be given.

The amounts to which the homework, mid-terms and final exam contributes to your grade are:
\begin{center}
\begin{tabular}{lc}
Mid-term Exams & 45\%\\
Final Exam     & 45\%\\
Assignments    & 10\%\\
\end{tabular}
\end{center}

Letter grades will be computed from your overall numerical score at mid-semester and again at the end of the semester.
%Grades will be computed on an absolute scale, not on a curve.
You are not in competition with your fellow students, but only with yourself and with the material.
%The final letter grades for the course will be determined by the following approximate scale:
%\begin{center}
%\begin{tabular}{ccccc}
%  A: $>$87\% &  B: 75-87\% &  C: 60-75\% &  D: 50-60\% &  R: $<$50\%
%\end{tabular}
%\end{center}
In borderline cases, based on the instructors' perception of your work, we will consider diligent lecture attendance, class participation, and consistency of performance throughout the semester.



\section{Assignment Philosophy:}

There will be weekly homework assignments, consisting of readings and written problems.
Doing the assignments, especially problem solving, is the single most important tool for learning the course material.
The assignments are intended to:\\

\begin{itemize}
\item[-] Fill in claims/sketches of arguments made in class.
\item[-] Reiterate concepts discussed in class.
\item[-] Build your problem solving skills
\item[-] Introduce/exercise ``straight-forward'' concepts from your reading. 
\item[-] Set the stage for future topics.
\item[-] ``Fun facts'': ideas/calculations that are interesting (or historically important) that you can now understand.
\end{itemize}



Homework dealing with the ongoing week's concepts will be due on Fridays, uploaded to Canvas.
Written solutions will later be provided on the course website.
A random subset of the problems will be graded in detail, and all problems will be checked to see if you made a serious effort to solve them.
20\% credit per day will be deducted for late work, and zero credit will be given after the solutions are posted.

\textit{Pro-Tip:} review your returned work and the posted solutions each week, even the problems that were not graded in detail. Test problems will be similar to the homework problems.



\section{Assignment Collaboration Policy:} 

Speak to each other!
You often learn just as effectively through discussion with your peers as you do from lecture. 
However, all homework you submit should be written individually and independently by you.
%When it comes to discussing homework, any records of the discussion must be destroyed before you make any kind of write-up. 
%If you have collaborated with anyone then you should declare who you worked with and the nature of your discussion. 
%\textit{(Think citation for a publication)}
%You will not be penalized for \underline{declared} homework collaboration. 
%Undeclared collaboration is plagiarism and is considered cheating. 
All the work you submit should be your own work and reflect your understanding.

\clearpage

\section{Preliminary Schedule:}

%
% MD: 3 weeks on kinematics
%

%
% RS:
%


The following is a rough outline of topics we will discuss each week. 

\begin{tabular}{llll}
\textbf{Week 1}   & Big picture / Foundations of Relativity    & Jan -/18/19/20 & \\
\textbf{Week 2}   & The interval / Lorentz Transformations     & Jan 23/25/26/27 & \\
\textbf{Week 3}   & Spacetime diagrams / Doppler Effect        & Jan 30/ Feb 1/2/3 & \\
\textbf{Week 4}   & Paradoxes / \textbf{Exam 1  \#1 Friday}    & Feb 6/8/9/\textbf{10} & \\
\textbf{Week 5}   & Dynamics / Invariant Mass                  & Feb 13/15/16/17 & \\
\textbf{Week 6}   & Dynamics                                   & Feb 20/22/23/24 & \\
\textbf{Week 7}   &           / \textbf{Exam \#2 Friday}       & Feb 27/March 1/2/ \textbf{3} & \\
\textbf{Week off} &  Spring Break                              & -/-/-/- & \\
\textbf{Week 8}   &  Unsettling story / Quantization of energy & March 13/15/16/17 & \\
\textbf{Week 9}   &  Quantization of Light                     & March 20/22/23/24 & \\
\textbf{Week 10}  &  Atoms                                     & March 27/29/30/31 & \\
\textbf{Week 11}  &  Wave nature of particles                  & April 3/5/6/7 & \\
\textbf{Week 12}  &  Uncertainty Principle /  \textbf{Exam \#3 Wednesday}           & April 10/\textbf{12}/-/- & \\
\textbf{Week 13}  &  Schrodinger Equation                      & April 17/19/20/21 & \\
\textbf{Week 14}  &  Atoms done right                          & April 24/26/27/28 & \\
\end{tabular}

\end{resume}
\end{document}



