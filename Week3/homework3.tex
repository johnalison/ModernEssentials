\documentclass[paper=letter,11pt]{scrartcl}

\KOMAoptions{headinclude=true, footinclude=false}
\KOMAoptions{DIV=14, BCOR=5mm}
\KOMAoptions{numbers=noendperiod}
\KOMAoptions{parskip=half}
\addtokomafont{disposition}{\rmfamily}
\addtokomafont{part}{\LARGE}
\addtokomafont{descriptionlabel}{\rmfamily}
%\setkomafont{pageheadfoot}{\normalsize\sffamily}
\setkomafont{pagehead}{\normalsize\rmfamily}
%\setkomafont{publishers}{\normalsize\rmfamily}
\setkomafont{caption}{\normalfont\small}
\setcapindent{0pt}
\deffootnote[1em]{1em}{1em}{\textsuperscript{\thefootnotemark}\ }


\usepackage{amsmath}
\usepackage[varg]{txfonts}
\usepackage[T1]{fontenc}
\usepackage{graphicx}
\usepackage{xcolor}
\usepackage[american]{babel}
% hyperref is needed in many places, so include it here
\usepackage{hyperref}

\usepackage{xspace}
\usepackage{multirow}
\usepackage{float}


\usepackage{braket}
\usepackage{bbm}
\usepackage{relsize}
\usepackage{tcolorbox}


\def\x{\ensuremath{x}}
\def\xp{\ensuremath{x'}}
\def\t{\ensuremath{t}}
\def\tp{\ensuremath{t'}}
\def\v{\ensuremath{v}}
%\def\nus{$\nu$s}

%\def\ketY{\ensuremath{\ket {\Psi}}}
%\def\iGeV{\ensuremath{\textrm{GeV}^{-1}}}
%%\def\mp{\ensuremath{m_{\textrm{proton}}}}
%\def\rp{\ensuremath{r_{\textrm{proton}}}}
%\def\me{\ensuremath{m_{\textrm{electron}}}}
%\def\aG{\ensuremath{\alpha_G}}
%\def\rAtom{\ensuremath{r_{\textrm{atom}}}}
%\def\rNucl{\ensuremath{r_{\textrm{nucleus}}}}
%\def\GN{\ensuremath{\textrm{G}_\textrm{N}}}
%\def\ketX{\ensuremath{\ket{\vec{x}}}}
%\def\ve{\ensuremath{\vec{\epsilon}}}
%
%
%\def\ABCDMatrix{\ensuremath{\begin{pmatrix} A &  B  \\ C  & D \end{pmatrix}}}
%\def\xyprime{\ensuremath{\begin{pmatrix} x' \\ y' \end{pmatrix}}}
%\def\xyprimeT{\ensuremath{\begin{pmatrix} x' &  y' \end{pmatrix}}}
%\def\xy{\ensuremath{\begin{pmatrix} x \\ y \end{pmatrix}}}
%\def\xyT{\ensuremath{\begin{pmatrix} x & y \end{pmatrix}}}
%
%\def\IMatrix{\ensuremath{\begin{pmatrix} 0 &  1  \\ -1  & 0 \end{pmatrix}}}
%\def\IBoostMatrix{\ensuremath{\begin{pmatrix} 0 &  1  \\ 1  & 0 \end{pmatrix}}}
%\def\JThree{\ensuremath{\begin{pmatrix}    0 & -i & 0  \\ i & 0  & 0 \\ 0 & 0 & 0 \end{pmatrix}}} 
%\def\JTwo{\ensuremath{\begin{bmatrix}    0 & 0 & -i  \\ 0 & 0  & 0 \\ i & 0 & 0 \end{bmatrix}}}
%\def\JOne{\ensuremath{\begin{bmatrix}    0 & 0 & 0  \\ 0 & 0  & -i \\ 0 & i & 0 \end{bmatrix}}}
%\def\etamn{\ensuremath{\eta_{\mu\nu}}}
%\def\Lmn{\ensuremath{\Lambda^\mu_\nu}}
%\def\dmn{\ensuremath{\delta^\mu_\nu}}
%\def\wmn{\ensuremath{\omega^\mu_\nu}}
%\def\be{\begin{equation*}}
%\def\ee{\end{equation*}}
%\def\bea{\begin{eqnarray*}}
%\def\eea{\end{eqnarray*}}
%\def\bi{\begin{itemize}}
%\def\ei{\end{itemize}}
%\def\fmn{\ensuremath{F_{\mu\nu}}}
%\def\fMN{\ensuremath{F^{\mu\nu}}}
%\def\bc{\begin{center}}
%\def\ec{\end{center}}
%\def\nus{$\nu$s}

\def\adagger{\ensuremath{a_{p\sigma}^\dagger}}
\def\lineacross{\noindent\rule{\textwidth}{1pt}}

\newcommand{\multiline}[1] {
\begin{tabular} {|l}
#1
\end{tabular}
}

\newcommand{\multilineNoLine}[1] {
\begin{tabular} {l}
#1
\end{tabular}
}



\newcommand{\lineTwo}[2] {
\begin{tabular} {|l}
#1 \\
#2
\end{tabular}
}

\newcommand{\rmt}[1] {
\textrm{#1}
}


%
% Units
%
\def\m{\ensuremath{\rmt{m}}}
\def\GeV{\ensuremath{\rmt{GeV}}}
\def\pt{\ensuremath{p_\rmt{T}}}


\def\parity{\ensuremath{\mathcal{P}}}

\usepackage{cancel}
\usepackage{ mathrsfs }
\def\bigL{\ensuremath{\mathscr{L}}}

\usepackage{ dsfont }

\def\nus{$\nu$s}
\def\nue{\ensuremath{\nu_e}}
\def\numu{\ensuremath{\nu_\mu}}
\def\nutau{\ensuremath{\nu_\tau}}
\def\nualpha{\ensuremath{\nu_\alpha}}
\def\nuone{\ensuremath{\nu_1}}
\def\nutwo{\ensuremath{\nu_2}}
\def\nuthree{\ensuremath{\nu_3}}


\usepackage{fancyhdr}
\fancyhf{}


\def\xyprime{\ensuremath{\begin{pmatrix} x' \\ y' \end{pmatrix}}}


\lhead{\Large 33-211} % \hfill Introduction to Particle Physics \hfill Spring 2022}
\chead{\Large Physics 3 : Modern Essentials} % \hfill Spring 2022}
\rhead{\Large Spring 2023} % \hfill Introduction to Particle Physics \hfill Spring 2022}
\begin{document}
\thispagestyle{fancy}





%\begin{tabular}{c}
%{\large 33-444 \hfill Intro To Particle \hfill Spring 2022\\}
%\hline 
%\end{tabular}

\begin{center}
{\huge \textbf{Homework Set \#2}}
\large

{\textbf{ Due Date:} Before class Friday February 3rd  }
\end{center}

\textbf{1) Reading } \hfill \textit{(2 points)}\\
Finish chapter 1 and re-read the sections from homework \#1.

\vspace*{0.25in}


{\large
\textbf{2) Michelson-Morley experiment } \hfill \textit{(8 points)}\\
Show that the null effect of the Michelson-Morley experiment can be accounted for by Lorentz contraction.

\vspace*{0.25in}


\textbf{3) Moving Balls } \hfill \textit{(2 points)}\\
Imagine  a spherical ball  object at rest in the S' frame.
Let S' be moving to the right with respect to the S frame at high speed $\beta$.
According to the principle of relativity, observers in S will measure the object to be: (list all that are true) 

\begin{itemize}
\item[-] a) Spherical in shape, but reduced in radius by the stretch-factor $\gamma$
\item[-] b) Oblate (pancake shaped), with the short direction along x
\item[-] c) Prolate (cigar shaped), with the short direction transverse to x 
\end{itemize}

\vspace*{0.25in}

\textbf{4) Provisions in space  } \hfill \textit{(5 points)}\\
You are traveling to a star system that is 10 light years away.
The star-ship on which you travel moves at 0.5c.
You have trained to eat one meal a day.
How many meals do you need to bring?

\vspace*{0.25in}


\textbf{5) Velocity Addition  } \hfill \textit{(5 points)}\\
A particle moves with speed 0.9c along the x'' axis of frame S'' which moves with speed 0.9c in the positive x' direction relative to frame S'.
Frame S' moves with speed 0.9c in the positive x direction relative to frame S. Find the speed of the particle relative to frame S.

\vspace*{0.25in}

\textbf{6) Break down of Galilean Transformations   } \hfill \textit{(5 points)}\\
How great must the relative speed of two observers be for their time-interval measurements to differ by 1 percent ?

\clearpage

\textbf{7) Temporal order  } \hfill \textit{(5 points)}\\
Show that the temporal order of two events in the laboratory frame is the same in all rocket frames if they event have time-like or like-light separation.

What goes wrong if they are space-like separated ?

\vspace*{0.25in}

\textbf{8) Lorentz Contraction  } \hfill \textit{(10 points)}\\
A meter stick lies along the x' axis and at rest in the rocket frame.
Show that an observer in the laboratory frame will conclude that the meter stick is Lorentz contracted if they measure how long it take the meter stick to pass a point in their reference frame and multiples the result by the relative velocity of the two frames.

\vspace*{0.25in}

\textbf{9) Time dilation  } \hfill \textit{(10 points)}\\
Two events occur at the same place but at different times in the rocket frame.
Show that an observer in the laboratory frame will conclude that the time between the two events has been dilated if he measures the distance between them in the laboratory frame and divides this distance by the relative velocity of the two frames.

\vspace*{0.25in}

\textbf{10) Euclidean ``slope'' transformations   } \hfill \textit{(10 points)}\\
Show that the coordinate transformation of our Euclidean geometry analogy is given by,
\begin{equation*}
\begin{pmatrix} x \\ y \end{pmatrix} = \begin{pmatrix} G & -BG \\ BG & G \end{pmatrix} \begin{pmatrix} x' \\ y' \end{pmatrix}
\end{equation*}
where is $B$ is slope of the x'-axis as measured in the S frame, and $G\equiv \frac{1}{\sqrt{1+B^2}}$. 


\end{document}
