\documentclass[paper=letter,11pt]{scrartcl}

\KOMAoptions{headinclude=true, footinclude=false}
\KOMAoptions{DIV=14, BCOR=5mm}
\KOMAoptions{numbers=noendperiod}
\KOMAoptions{parskip=half}
\addtokomafont{disposition}{\rmfamily}
\addtokomafont{part}{\LARGE}
\addtokomafont{descriptionlabel}{\rmfamily}
%\setkomafont{pageheadfoot}{\normalsize\sffamily}
\setkomafont{pagehead}{\normalsize\rmfamily}
%\setkomafont{publishers}{\normalsize\rmfamily}
\setkomafont{caption}{\normalfont\small}
\setcapindent{0pt}
\deffootnote[1em]{1em}{1em}{\textsuperscript{\thefootnotemark}\ }


\usepackage{amsmath}
\usepackage[varg]{txfonts}
\usepackage[T1]{fontenc}
\usepackage{graphicx}
\usepackage{xcolor}
\usepackage[american]{babel}
% hyperref is needed in many places, so include it here
\usepackage{hyperref}

\usepackage{xspace}
\usepackage{multirow}
\usepackage{float}


\usepackage{braket}
\usepackage{bbm}
\usepackage{relsize}
\usepackage{tcolorbox}


\def\x{\ensuremath{x}}
\def\xp{\ensuremath{x'}}
\def\t{\ensuremath{t}}
\def\tp{\ensuremath{t'}}
\def\v{\ensuremath{v}}
%\def\nus{$\nu$s}

%\def\ketY{\ensuremath{\ket {\Psi}}}
%\def\iGeV{\ensuremath{\textrm{GeV}^{-1}}}
%%\def\mp{\ensuremath{m_{\textrm{proton}}}}
%\def\rp{\ensuremath{r_{\textrm{proton}}}}
%\def\me{\ensuremath{m_{\textrm{electron}}}}
%\def\aG{\ensuremath{\alpha_G}}
%\def\rAtom{\ensuremath{r_{\textrm{atom}}}}
%\def\rNucl{\ensuremath{r_{\textrm{nucleus}}}}
%\def\GN{\ensuremath{\textrm{G}_\textrm{N}}}
%\def\ketX{\ensuremath{\ket{\vec{x}}}}
%\def\ve{\ensuremath{\vec{\epsilon}}}
%
%
%\def\ABCDMatrix{\ensuremath{\begin{pmatrix} A &  B  \\ C  & D \end{pmatrix}}}
%\def\xyprime{\ensuremath{\begin{pmatrix} x' \\ y' \end{pmatrix}}}
%\def\xyprimeT{\ensuremath{\begin{pmatrix} x' &  y' \end{pmatrix}}}
%\def\xy{\ensuremath{\begin{pmatrix} x \\ y \end{pmatrix}}}
%\def\xyT{\ensuremath{\begin{pmatrix} x & y \end{pmatrix}}}
%
%\def\IMatrix{\ensuremath{\begin{pmatrix} 0 &  1  \\ -1  & 0 \end{pmatrix}}}
%\def\IBoostMatrix{\ensuremath{\begin{pmatrix} 0 &  1  \\ 1  & 0 \end{pmatrix}}}
%\def\JThree{\ensuremath{\begin{pmatrix}    0 & -i & 0  \\ i & 0  & 0 \\ 0 & 0 & 0 \end{pmatrix}}} 
%\def\JTwo{\ensuremath{\begin{bmatrix}    0 & 0 & -i  \\ 0 & 0  & 0 \\ i & 0 & 0 \end{bmatrix}}}
%\def\JOne{\ensuremath{\begin{bmatrix}    0 & 0 & 0  \\ 0 & 0  & -i \\ 0 & i & 0 \end{bmatrix}}}
%\def\etamn{\ensuremath{\eta_{\mu\nu}}}
%\def\Lmn{\ensuremath{\Lambda^\mu_\nu}}
%\def\dmn{\ensuremath{\delta^\mu_\nu}}
%\def\wmn{\ensuremath{\omega^\mu_\nu}}
%\def\be{\begin{equation*}}
%\def\ee{\end{equation*}}
%\def\bea{\begin{eqnarray*}}
%\def\eea{\end{eqnarray*}}
%\def\bi{\begin{itemize}}
%\def\ei{\end{itemize}}
%\def\fmn{\ensuremath{F_{\mu\nu}}}
%\def\fMN{\ensuremath{F^{\mu\nu}}}
%\def\bc{\begin{center}}
%\def\ec{\end{center}}
%\def\nus{$\nu$s}

\def\adagger{\ensuremath{a_{p\sigma}^\dagger}}
\def\lineacross{\noindent\rule{\textwidth}{1pt}}

\newcommand{\multiline}[1] {
\begin{tabular} {|l}
#1
\end{tabular}
}

\newcommand{\multilineNoLine}[1] {
\begin{tabular} {l}
#1
\end{tabular}
}



\newcommand{\lineTwo}[2] {
\begin{tabular} {|l}
#1 \\
#2
\end{tabular}
}

\newcommand{\rmt}[1] {
\textrm{#1}
}


%
% Units
%
\def\m{\ensuremath{\rmt{m}}}
\def\GeV{\ensuremath{\rmt{GeV}}}
\def\pt{\ensuremath{p_\rmt{T}}}


\def\parity{\ensuremath{\mathcal{P}}}

\usepackage{cancel}
\usepackage{ mathrsfs }
\def\bigL{\ensuremath{\mathscr{L}}}

\usepackage{ dsfont }

\def\nus{$\nu$s}
\def\nue{\ensuremath{\nu_e}}
\def\numu{\ensuremath{\nu_\mu}}
\def\nutau{\ensuremath{\nu_\tau}}
\def\nualpha{\ensuremath{\nu_\alpha}}
\def\nuone{\ensuremath{\nu_1}}
\def\nutwo{\ensuremath{\nu_2}}
\def\nuthree{\ensuremath{\nu_3}}


\usepackage{fancyhdr}
\fancyhf{}


\def\xyprime{\ensuremath{\begin{pmatrix} x' \\ y' \end{pmatrix}}}


\lhead{\Large 33-211} % \hfill Introduction to Particle Physics \hfill Spring 2022}
\chead{\Large Physics 3 : Modern Essentials} % \hfill Spring 2022}
\rhead{\Large Spring 2023} % \hfill Introduction to Particle Physics \hfill Spring 2022}
\begin{document}
\thispagestyle{fancy}





%\begin{tabular}{c}
%{\large 33-444 \hfill Intro To Particle \hfill Spring 2022\\}
%\hline 
%\end{tabular}

\begin{center}
{\huge \textbf{Homework Set \#4}}
\large

{\textbf{ Due Date:} Before class Friday February 24th  }
\end{center}

\textbf{1) Reading } \hfill \textit{(2 points)}\\
Re-read chapter 2.

\vspace*{0.25in}


{\large
\textbf{2) Transformations of the Energy Momentum Four Vector } \hfill \textit{(10 points)}\\
Show that the Energy-Momentum 4-vector (E,$\vec{p}$) transforms like (t,$\vec{x}$) under a Lorentz Transform.
Explicitly calculate the transformations by transforming $\gamma$ and $\vec{\beta}$.

\textit{Hint first show that $\gamma_{\vec{\beta}} = \gamma_{\vec{\beta'}}\gamma_{\beta_S}(1+\beta_{S} \beta_{x}')$,
where $\gamma_{\vec{\beta}}$ is the particle $\gamma$ in the S frame,
      $\gamma_{\vec{\beta'}}$ is the particle $\gamma$ in the S' frame, $\beta_{x}'$ is the x-component of the particle speed in the S' frame, and $\beta_{S}$ is the relative velocity between frames in the x-direction}


\vspace*{0.25in}

\textbf{3) Mass Conversion Examples } \hfill \textit{(5 points)}\\
\begin{itemize}
\item[1] How much mass does a 100-W bulb dissipate in one year?
\item[2] The US generates $\sim 3 10^{12}$ kWh of energy/year. How much mass is this equivalent to ?
\item[3] A student pedaling a bicycle all out produces about 0.5 horsepower of useful power. The human body is about 25\% efficient.
         How long will our student have to ride (all out) to lose one pound by the conversion of mass to energy ? How do people loose weight at the gym?
\item[4] About 1.4 kW of sunlight falls on 1 $m^2$ of earths surface (ignoring the effects of the atmosphere). How much mass does the sun radiate as light in one second?
         How much of the suns mass reaches the earth per year in the form of light?
\item[5] Two trains, each weighing $10^8$ kg, traveling in opposite directions at 100 mph collide and come to rest. How much is the rest mass of the collision debris increased immediately after the collision?
\end{itemize}

\vspace*{0.25in}

\textbf{4) Inelastic collision } \hfill \textit{(10 points)}\\
Solve the total inelastic collision where two identical initial particles have mass m and collide head-on with speed $\beta$ in the center of mass frame to form a single combined system: $A + A \rightarrow B$ . 
Express the mass of the combined particle in terms of the mass of the initial particles and the $\beta$.\\
What is the final energy in S ?

Now solve the problem in the frames S' that moves wrt S with speed $\beta$. 
Is energy the same in S and S' ?
Now relate the energies in S and S' directly with a Lorentz Transformation.


\textbf{5) Symmetric Elastic Collisions (Again) } \hfill \textit{(15 points)}\\
Reconsider problem \#8 from homework \#3 of proton A colliding elastically with proton B at rest and leading to a ``symmetric collision'' in which the two protons come off with identical speeds along paths that make identical angles $\alpha_A = \alpha_B = \alpha/2$ with the forward direction.
Let proton A have initial kinetic energy $KE$.  
Determine the angle $\alpha$ using the relativistic form of conservation of E and P. 
Show that the angle is given by
\begin{equation*}
\cos^2\left( \frac{\alpha}{2}\right) = \frac{KE + 2m_p}{KE + 4m_p}
\end{equation*}
and thus (from a trig id)
\begin{equation*}
\cos \alpha = \frac{KE}{ KE + 4m}
\end{equation*}
What is the angle $\alpha$ in the Newtonian regime? What about the extreme relativistic regime ?

\vspace*{0.25in}

\textbf{6) Asymmetric Collision  } \hfill \textit{(20 points)}\\
Consider Particle A of mass $M$ moving toward a collision with a Particle B of mass $4M$.
Particle A has momentum $\vec{p}_A=(3M,3M,0)$; Particle B has momentum $\vec{p}_B=(0,3M,0)$
\begin{itemize}
\item[1] What are the energies, kinetic energies and speeds of the particles before the collision?  
\item[2] What are the energies of the particles in the A rest frame?
\item[3] What are the energies of the particles in the B rest frame?
\item[4] What is the 3-momentum of the combined system?
\item[5] How does the mass of the system compare to $m_A + m_B$ ? Is it different ? Why/ why not?
\item[6] What is the speed of the combined system after a total inelastic collision (ie: the particles stick together) ?
\end{itemize}


\vspace*{0.25in}

\textbf{7) Photons  } \hfill \textit{(10 points)}\\
A photon moves in the xy plane in the S' frame in a direction that makes an angle $\phi'$ with the x-axis.
What angle does the photon make in the S frame?
What is the photon energy in the S frame. (S' moves along the x-direction with speed $\beta$ in the usual way).

\vspace*{0.25in}

\textbf{8) $\pi^0$ Meson }  \hfill \textit{(15 points)}\\
A $\pi^0$ meson is a subatomic particle that is composed of a quark-antiquark pair, bound together by the strong nuclear force.
A $\pi^0$ is moving in the x direction with a KE in the laboratory frame equal to its rest energy.
It decays into two photons.
In the rest from of the $\pi^0$ the photons are emitted in the $\pm y$-direction.
What are the energies and the directions of propagation of the two photons in the laboratory frame?

\end{document}
