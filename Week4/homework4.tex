\documentclass[paper=letter,11pt]{scrartcl}

\KOMAoptions{headinclude=true, footinclude=false}
\KOMAoptions{DIV=14, BCOR=5mm}
\KOMAoptions{numbers=noendperiod}
\KOMAoptions{parskip=half}
\addtokomafont{disposition}{\rmfamily}
\addtokomafont{part}{\LARGE}
\addtokomafont{descriptionlabel}{\rmfamily}
%\setkomafont{pageheadfoot}{\normalsize\sffamily}
\setkomafont{pagehead}{\normalsize\rmfamily}
%\setkomafont{publishers}{\normalsize\rmfamily}
\setkomafont{caption}{\normalfont\small}
\setcapindent{0pt}
\deffootnote[1em]{1em}{1em}{\textsuperscript{\thefootnotemark}\ }


\usepackage{amsmath}
\usepackage[varg]{txfonts}
\usepackage[T1]{fontenc}
\usepackage{graphicx}
\usepackage{xcolor}
\usepackage[american]{babel}
% hyperref is needed in many places, so include it here
\usepackage{hyperref}

\usepackage{xspace}
\usepackage{multirow}
\usepackage{float}


\usepackage{braket}
\usepackage{bbm}
\usepackage{relsize}
\usepackage{tcolorbox}


\def\x{\ensuremath{x}}
\def\xp{\ensuremath{x'}}
\def\t{\ensuremath{t}}
\def\tp{\ensuremath{t'}}
\def\v{\ensuremath{v}}
%\def\nus{$\nu$s}

%\def\ketY{\ensuremath{\ket {\Psi}}}
%\def\iGeV{\ensuremath{\textrm{GeV}^{-1}}}
%%\def\mp{\ensuremath{m_{\textrm{proton}}}}
%\def\rp{\ensuremath{r_{\textrm{proton}}}}
%\def\me{\ensuremath{m_{\textrm{electron}}}}
%\def\aG{\ensuremath{\alpha_G}}
%\def\rAtom{\ensuremath{r_{\textrm{atom}}}}
%\def\rNucl{\ensuremath{r_{\textrm{nucleus}}}}
%\def\GN{\ensuremath{\textrm{G}_\textrm{N}}}
%\def\ketX{\ensuremath{\ket{\vec{x}}}}
%\def\ve{\ensuremath{\vec{\epsilon}}}
%
%
%\def\ABCDMatrix{\ensuremath{\begin{pmatrix} A &  B  \\ C  & D \end{pmatrix}}}
%\def\xyprime{\ensuremath{\begin{pmatrix} x' \\ y' \end{pmatrix}}}
%\def\xyprimeT{\ensuremath{\begin{pmatrix} x' &  y' \end{pmatrix}}}
%\def\xy{\ensuremath{\begin{pmatrix} x \\ y \end{pmatrix}}}
%\def\xyT{\ensuremath{\begin{pmatrix} x & y \end{pmatrix}}}
%
%\def\IMatrix{\ensuremath{\begin{pmatrix} 0 &  1  \\ -1  & 0 \end{pmatrix}}}
%\def\IBoostMatrix{\ensuremath{\begin{pmatrix} 0 &  1  \\ 1  & 0 \end{pmatrix}}}
%\def\JThree{\ensuremath{\begin{pmatrix}    0 & -i & 0  \\ i & 0  & 0 \\ 0 & 0 & 0 \end{pmatrix}}} 
%\def\JTwo{\ensuremath{\begin{bmatrix}    0 & 0 & -i  \\ 0 & 0  & 0 \\ i & 0 & 0 \end{bmatrix}}}
%\def\JOne{\ensuremath{\begin{bmatrix}    0 & 0 & 0  \\ 0 & 0  & -i \\ 0 & i & 0 \end{bmatrix}}}
%\def\etamn{\ensuremath{\eta_{\mu\nu}}}
%\def\Lmn{\ensuremath{\Lambda^\mu_\nu}}
%\def\dmn{\ensuremath{\delta^\mu_\nu}}
%\def\wmn{\ensuremath{\omega^\mu_\nu}}
%\def\be{\begin{equation*}}
%\def\ee{\end{equation*}}
%\def\bea{\begin{eqnarray*}}
%\def\eea{\end{eqnarray*}}
%\def\bi{\begin{itemize}}
%\def\ei{\end{itemize}}
%\def\fmn{\ensuremath{F_{\mu\nu}}}
%\def\fMN{\ensuremath{F^{\mu\nu}}}
%\def\bc{\begin{center}}
%\def\ec{\end{center}}
%\def\nus{$\nu$s}

\def\adagger{\ensuremath{a_{p\sigma}^\dagger}}
\def\lineacross{\noindent\rule{\textwidth}{1pt}}

\newcommand{\multiline}[1] {
\begin{tabular} {|l}
#1
\end{tabular}
}

\newcommand{\multilineNoLine}[1] {
\begin{tabular} {l}
#1
\end{tabular}
}



\newcommand{\lineTwo}[2] {
\begin{tabular} {|l}
#1 \\
#2
\end{tabular}
}

\newcommand{\rmt}[1] {
\textrm{#1}
}


%
% Units
%
\def\m{\ensuremath{\rmt{m}}}
\def\GeV{\ensuremath{\rmt{GeV}}}
\def\pt{\ensuremath{p_\rmt{T}}}


\def\parity{\ensuremath{\mathcal{P}}}

\usepackage{cancel}
\usepackage{ mathrsfs }
\def\bigL{\ensuremath{\mathscr{L}}}

\usepackage{ dsfont }

\def\nus{$\nu$s}
\def\nue{\ensuremath{\nu_e}}
\def\numu{\ensuremath{\nu_\mu}}
\def\nutau{\ensuremath{\nu_\tau}}
\def\nualpha{\ensuremath{\nu_\alpha}}
\def\nuone{\ensuremath{\nu_1}}
\def\nutwo{\ensuremath{\nu_2}}
\def\nuthree{\ensuremath{\nu_3}}


\usepackage{fancyhdr}
\fancyhf{}


\def\xyprime{\ensuremath{\begin{pmatrix} x' \\ y' \end{pmatrix}}}


\lhead{\Large 33-211} % \hfill Introduction to Particle Physics \hfill Spring 2022}
\chead{\Large Physics 3 : Modern Essentials} % \hfill Spring 2022}
\rhead{\Large Spring 2023} % \hfill Introduction to Particle Physics \hfill Spring 2022}
\begin{document}
\thispagestyle{fancy}





%\begin{tabular}{c}
%{\large 33-444 \hfill Intro To Particle \hfill Spring 2022\\}
%\hline 
%\end{tabular}

\begin{center}
{\huge \textbf{Homework Set \#4}}
\large

{\textbf{ Due Date:} Before class Friday February 24th  }
\end{center}

\textbf{1) Reading } \hfill \textit{(2 points)}\\
Re-read chapter 2.

\vspace*{0.25in}


{\large
\textbf{2) Transformations of the Energy Momentum Four Vector } \hfill \textit{(8 points)}\\
Show that the Energy-Momentum four vector (E,$\vec{p}$) transforms as (t,$\vec{x}$) under a Lorentz Transform.
Explicitly calculate the transformations by transforming $\gamma$ and $\vec{Beta}$.

\textit{Hint first show that $\gamma_{\vec{\beta}} = \gamma_{\vec{\beta'}}\gamma_{\vec{\beta}}(1+\beta_{S'} \beta_x)$, where ... }


\vspace*{0.25in}

\textbf{3) Conversion examples 62 and 63 } \hfill \textit{(8 points)}\\


\textbf{4) Inelasitic collision } \hfill \textit{(8 points)}\\
Solve the total inelastic collision $A + A \rightarrow B$ where the two initial particles have mass m and collide head-on with speed $\beta$ in the cetnter of mass frame. 
What is your expression for the mass of B?
Now solve the problem in the frams S' that move wrt S with speed $\beta$. 
What is the final energy in S ? Is energy the same in S and S' ?
Now relate the energies in S and S' directly with a Lorentz Transformation.


\textbf{5) Symmetric Elastic Collisions Again } \hfill \textit{(8 points)}\\
Deterime the angle using teh relativistic form of the conservation of E and P 

show that the angle is given by cos^2 alpha / 2 = T + 2m/T+4b
and thus (from a trig id)
cos alpha = T / T+ 4m
What is the angle for the Newtonian regime? Waht ais the angle for teh extreem relativistic regime?



\textbf{6) Antiprotons 93

\textbf{7) Storage rings 100

\textbf{8) photon energy, angles 75 and 76




\end{document}
